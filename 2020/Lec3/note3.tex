\documentclass[10pt,a4j]{article}
%\usepackage{graphicx,wrapfig}
\usepackage{graphicx}
\setlength{\topmargin}{-1.5cm}
%\setlength{\leftmargin}{1.5cm}
%\setlength{\textwidth}{15.5cm}
\setlength{\textheight}{25.2cm}
\newlength{\minitwocolumn}
\setlength{\minitwocolumn}{0.5\textwidth}
\addtolength{\minitwocolumn}{-\columnsep}
%\addtolength{\baselineskip}{-0.1\baselineskip}
%
\def\Mmaru#1{{\ooalign{\hfil#1\/\hfil\crcr
\raise.167ex\hbox{\mathhexbox 20D}}}}
%
\begin{document}
\newcommand{\fat}[1]{\mbox{\boldmath $#1$}}
\newcommand{\D}{\partial}
\newcommand{\w}{\omega}
\newcommand{\ga}{\alpha}
\newcommand{\gb}{\beta}
\newcommand{\gx}{\xi}
\newcommand{\gz}{\zeta}
\newcommand{\vhat}[1]{\hat{\fat{#1}}}
\newcommand{\spc}{\vspace{0.7\baselineskip}}
\newcommand{\halfspc}{\vspace{0.3\baselineskip}}
\bibliographystyle{unsrt}
%\pagestyle{empty}
\newcommand{\twofig}[2]
 {
   \begin{figure}
     \begin{minipage}[t]{\minitwocolumn}
         \begin{center}   #1
         \end{center}
     \end{minipage}
         \hspace{\columnsep}
     \begin{minipage}[t]{\minitwocolumn}
         \begin{center} #2
         \end{center}
     \end{minipage}
   \end{figure}
 }
%%%%%%%%%%%%%%%%%%%%%%%%%%%%%%%%%
%\vspace*{\baselineskip}
\begin{flushright}
	25/04/2019
\end{flushright}
\begin{center}
	{\Large \bf Civil Engineering I \\
	- Introduction to structural mechanics-\\
	Application of equilibrium equations 
	}  \\
\end{center}
\vspace{1.5cm}
\section*{Exercises 3}
\begin{itemize}
\item
Exercise 3.1\\
A straight bar AC of length $l$ is subjected to the forces of magnitude  
$H_A, V_A, V_C$ and $|\fat{f}|=F$ as shown in Fig.\ref{fig:fig3_1}. 
For given $F$, determine $H_A, V_A$ and $V_C$ assuming that the bar is in a static equilibrium.,
\begin{figure}[h]
	\begin{center}
	\includegraphics[width=0.6\linewidth]{fig3_1.eps} 
	\end{center}
	\caption{A straight bar AC in a static equilibrium. 
	$F, H_A, V_A, V_C$ and $|\fat{f}|=F$ denote the magnitude of applied forces.
	}
	\label{fig:fig3_1}
\end{figure}
\item
Exercise 3.2\\
A straight bar AC is supported statically as shown in Fig.\ref{fig:fig3_2}. 
Determine the reaction forces from the supports at A and B. 
\begin{figure}[h]
	\begin{center}
	\includegraphics[width=0.6\linewidth]{fig3_2.eps} 
	\end{center}
	\caption{A simply supported bar AC subjected to an oblique force of magnitude $F$.}
	\label{fig:fig3_2}
\end{figure}
\item
Exercise 3.3\\
Consider a straight bar AB clamped rigidly by a wall as shown in Fig.\ref{fig:fig3_3}.
Determine the reaction forces and moment at the fixed end A  when 
a point force of magnitude $F$ is applied obliquely to the right end B of the bar.  
\begin{figure}[h]
	\begin{center}
	\includegraphics[width=0.6\linewidth]{fig3_3.eps} 
	\end{center}
	\caption{A bar AB connected to a fixed support. 
	An oblique load of magnitude $F$ is applied to the right end of the bar.} 
	\label{fig:fig3_3}
\end{figure}
\item
Exercise 3.4\\
Examine the support conditions shown in Fig.\ref{fig:fig3_4}, and answer whether 
all the reaction forces and moments may be determined using only the equilibrium equations. 
\begin{figure}[h]
	\begin{center}
	\includegraphics[width=0.8\linewidth]{fig3_4.eps} 
	\end{center}
	\caption{A straight bar connected to (a)fixed supports, and (b) pinned supports.} 
	\label{fig:fig3_4}
\end{figure}
\item
Exercise 3.5\\
Determine the reaction forces acting to the truss structures shown in Fig.\ref{fig:fig3_5} 
due to the application of the external forces of magnitude $F$.
\item
Exercise 3.6\\
Determine the axial forces generated in the truss structures shown in Fig.\ref{fig:fig3_5}
by the externally applied forces of magnitude $F$.
\begin{figure}[h]
	\begin{center}
	\includegraphics[width=1.0\linewidth]{fig3_5.eps} 
	\end{center}
	\caption{Simply supported truss structures.} 
	\label{fig:fig3_5}
\end{figure}
\end{itemize}
\end{document}
%%%%%%%%%%%%%%%%%%%%%%%%%%%%%%%%%%%%%%%%%%%%%%%%%%%%%%%%%%%%%%%%
