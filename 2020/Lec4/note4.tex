\documentclass[10pt,a4j]{article}
%\usepackage{graphicx,wrapfig}
\usepackage{graphicx}
%\usepackage{showkeys}
\setlength{\topmargin}{-1.5cm}
%\setlength{\leftmargin}{1.5cm}
%\setlength{\textwidth}{15.5cm}
\setlength{\textheight}{25.2cm}
\newlength{\minitwocolumn}
\setlength{\minitwocolumn}{0.5\textwidth}
\addtolength{\minitwocolumn}{-\columnsep}
%\addtolength{\baselineskip}{-0.1\baselineskip}
%
\def\Mmaru#1{{\ooalign{\hfil#1\/\hfil\crcr
\raise.167ex\hbox{\mathhexbox 20D}}}}
%
\begin{document}
\newcommand{\fat}[1]{\mbox{\boldmath $#1$}}
\newcommand{\D}{\partial}
\newcommand{\w}{\omega}
\newcommand{\ga}{\alpha}
\newcommand{\gb}{\beta}
\newcommand{\gx}{\xi}
\newcommand{\gz}{\zeta}
\newcommand{\vhat}[1]{\hat{\fat{#1}}}
\newcommand{\spc}{\vspace{0.7\baselineskip}}
\newcommand{\halfspc}{\vspace{0.3\baselineskip}}
\bibliographystyle{unsrt}
%\pagestyle{empty}
\newcommand{\twofig}[2]
 {
   \begin{figure}
     \begin{minipage}[t]{\minitwocolumn}
         \begin{center}   #1
         \end{center}
     \end{minipage}
         \hspace{\columnsep}
     \begin{minipage}[t]{\minitwocolumn}
         \begin{center} #2
         \end{center}
     \end{minipage}
   \end{figure}
 }
%%%%%%%%%%%%%%%%%%%%%%%%%%%%%%%%%
%\vspace*{\baselineskip}
\begin{flushright}
	Civil Engineering I\\
	May 14, 2020
\end{flushright}
\begin{center}
	{\LARGE \bf Lecuture Note 4 }
\end{center}
%\vspace{1.5cm}
\setcounter{section}{3}
\section{Design of Truss Structure}
\subsection{Axial Stress}
%\hspace{\parindent}
To design a truss structure, we need to define the strength of the structures. A straightforward way of defining the structural strength is to look at the maximum axial force that the truss members can bear without rupturing or yielding. If we define the structural strength by the allowable axial force, the definition is simple, clear, and quantitative. Furthermore, the structural strength in that sense can be evaluated by the structural analysis and a laboratory test on the loading capacity of the truss members.\\ 
The strength of a truss member depends on that of the material. Usually, the materials' strength is measured by a quantity called mechanical stress, which we denote sigma. Mechanical stress is an intensity of the force acting to a surface. In the context of this lecture, is defined by
\begin{equation}
	\sigma=\frac{N}{A}
	\label{eqn:def_sigma}
\end{equation}
where $N$ is the axial force and $A$ is the cross sectional area of a truss member. 
The dimension of the axial stress is force per unit area. It follows from the definition of the axial force that positive stress means a tensile loading while negative stress compressional loading. By performing an axial loading test on a truss member, we can identify the maximum tensile and compression axial stress
that the truss member can safely bear. We write the maximum compression and tensile stress values as $\sigma_{a}^c$ and $\sigma_{a}^t$. The two stress values are collectively called allowable stress.
The tensile and compressional allowable stresses usually do not coincide. For the sake of simplicity, we assume the two allowable stresses agree to each other and write
\begin{equation}
	\sigma_a^c=\sigma_a^t=\sigma_a.
	\label{eqn:sig_a}
\end{equation}
In designing a truss structure, we shall require 
\begin{equation}
	\left| \sigma \right| \leq \sigma_a
	\label{eqn:sig_a_cond}
\end{equation}
for every member of the truss. 
\subsection{Design of Simply Supported Truss -an example-}
Consider designing a simply supported Warren truss in bridging a trench of width 3$l$ (see Fig.\ref{fig:fig4_1}. 
To fabricate the truss, we use the bars of an equal length $l$ with a uniform circular 
cross-section of diameter $d$ as illustrated in Fig.\ref{fig:fig4_2}-(a).
When completed, the truss is used to support two straight water pipes (pipe 1 and 2) 
of identical cross-sections. The truss provides extra rigidity with the pipes and prevents the water-filled pipes from bending excessively. 
We design the truss used for this purpose based on the allowable stress concept. 
In the structural analysis necessary for the design, we assume the following.
\subsubsection{Assumptions}
\begin{enumerate}
\item
	The two pipes are identical and are filled with water. 
\item
	The truss members are so thin that their weight is negligible.
\item
	The weight of the water-filled pipes is supported evenly at the lower nodal 
		points (joints) A, A' to D,D' of the truss.
\item
	Due to the symmetry of the structure, the vertical force of magnitude 
	$F$ applied to the nodes A, B, C, and D comes only from pipe 1.
\item
	The extent of the pipe that needed to be supported by the truss is $4l$. 
	You can consider that the remainder of the weight goes to the supports on the 
	soil that are not shown in Fig.\ref{fig:fig4_1}.
\item
	A structural member breaks (yields) if the axial stress exceeds given allowable stress $\sigma_a$.
\end{enumerate}
\begin{figure}
	\begin{center}
	\includegraphics[width=1.25\linewidth]{fig4_1.eps} 
	\end{center}
	\caption{
		(a)Front, (b)side, and (c)top views of a simply supported Warren truss consisting 
		of bars of uniform circular cross-sections. 
		The truss bears the weight of two identical, symmetrically placed, water-filled pipes.
		The weight is distributed evenly to the nodal points (joints) on the lower level 
		applying the downward vertical load of magnitude $F$ to the nodes A, B, C and D.
	}
	\label{fig:fig4_1}
\end{figure}
\begin{figure}
	\begin{center}
	\includegraphics[width=0.5\linewidth]{fig4_2.eps} 
	\end{center}
	\caption{
		The cross-sections of a (a)truss member and (b)pipe.  
	}
	\label{fig:fig4_2}
\end{figure}
\subsubsection{Flow of Structural Analysis}
With the assumptions made above, we determine the minimum value of $d$ for which the 
truss can safely support the pipes without damaging (breaking) any of its members. 
The flow of structural analysis is as follows.
\begin{enumerate}
\item
	Determine the reaction forces applied to the truss by the supports at A and D. 
\item
	Obtain the axial forces of the truss member 1 through 11. 
\item
	Determine the maximum tensile and compression axial forces that arise in the truss members. 
\item
	Write the magnitude $F$ of the force using the dimensions $l,D,t$ of the bar and the pipe. 
	You may also use gravity constant $g$, the mass densities of water $\rho_w$ and steel $\rho_s$ 
	indicated in Fig.\ref{fig:fig4_2}. 
\item
	For given allowable stress $\sigma_a$, determine the minimum cross-sectional 
	area $A_{min}$ of the truss member and the corresponding diameter $d_{min}$. 
\item
	To get a better idea of what we have designed, evaluate $d_{min}$ numerically for\\
		\begin{center}
			$D$=500.0[mm], $t$=10.0[mm],  \\
			$\rho_s=$7.9[g/cm$^3$],
			$\rho_w=$1.0[g/cm$^3$],
			$g$=10.0[m/s$^2$] \\
			$l=3.0$[m],
			$\sigma_a$=20.0[kgf/mm$^2$], 
		\end{center}
	If necessary, you may use a scientific calculator, spread sheet application, or computer programming language. 
\end{enumerate}
\subsubsection{Design calculation}
Following the procedure in the previous section, the design calculation goes as follows.  
\begin{enumerate}
\item
	If we assume that $F$ is given, the equilibrium conditions for the hole truss 
	structure are written as follows.
	\begin{equation}
		H_A=0, \ \ V_A+V_D=4F, 
		\label{eqn:equib_fs}
	\end{equation}
	\begin{equation}
		V_D\times 3l - F\times l-F\times 2l-F\times 3l=0.
		\label{eqn:equib_Ma}
	\end{equation}
	Here, the total moment about node $A$ has been evaluated in eq.(\ref{eqn:equib_Ma}).
	For the free body diagram and the notations, see Fig.\ref{fig:fig1}-(a).
	Thus we have 
	\begin{equation}
		V_A=V_B=2F, \ \ H_A=0.
		\label{eqn:reac_fs}
	\end{equation}
\begin{figure}[h]
	\begin{center}
	\includegraphics[width=1.0\linewidth]{fig1.eps} 
	\end{center}
	\caption{(a)The free body diagram of the truss shown in Fig.\ref{fig:fig4_1}. 
	(b) Forces applied to the nodes of the truss structure.} 
	\label{fig:fig1}
\end{figure}
\item
	For the analysis of axial forces, we assign a unique number to 
	each truss member as shown in Fig.\ref{fig:fig1}-(b).
	The axial force of the $i$th truss member is denoted as $N_i$ as shown in this figure.
	Before we start imposing the equilibrium conditions node by node, it is worth examining 
	the symmetry of the structure. 
	Since $H_A=0$, the truss and the loading condition are symmetric about the vertical 
	line $a-a'$ shown in Fig.\ref{fig:fig1}-(b). 
	Therefore we have 	
	\[
		N_1=N_3, \ \ N_4=N_9,\ \  N_6=N_7
	\]
	reducing the number of unknown axial forces almost by half.
	Among seven nodes ($A,B,\dots D$), node $A$ and $D$ would produce 
	equilibrium equations solvable for the axial forces. So, we start imposing 
	equilibrium conditions from node $A$, and move on to node $E$ and $B$ successively. 
	As a result, we finally have the following. 
	\[
		N_1=N_3=\frac{1}{\sqrt{3}}F ,\ \ N_2=\frac{2}{\sqrt{3}}F
	\]
	\[
		N_4=N_9=-\frac{2}{\sqrt{3}}F ,\ \ N_5=N_8=\frac{2}{\sqrt{3}}F
	\]
	\[
		N_6=N_7=0, \ \ N_{10}=N_{11}=-\frac{2}{\sqrt{3}}F
	\]
\item
	The maximum compressional and tensile axial forces are $-\frac{2}{\sqrt{3}}F$ 
	and $\frac{2}{\sqrt{3}}F$, respectively. Hence, in aboslute value, the maximum axial force is $\frac{2}{\sqrt{3}}F$.
\item
	Let the weight of the pipe per length be denoted as $w$. Then $w$ is written as 
	\begin{equation}
		w= \left( \rho_w A_w + \rho_s A_\rho \right)g
		\label{eqn:}
	\end{equation}
	where $A_w$ and $A_w$ are the cross-sectional areas occupied by 
	water and the pipe wall, respectively. Using the diameter and the pipe-wall thickness, 
	the areas are written as 
	\begin{equation}
		A_w=\frac{\pi}{4}(D-2t)^2, \ \ A_s=\frac{\pi}{4}\left(D^2-(D-2t)^2\right).
		\label{eqn:}
	\end{equation}
	Since weight $F$ comes from a segment of length $l$ of the pipe , we have
	\begin{equation}
		F=wl=
		\frac{\pi}{4}gl
		\left\{ 
			\rho_w(D-2t)^2+ 4\rho_s\left(Dt-t^2\right)
		\right\}.
		\label{eqn:F_explicit}
	\end{equation}
\item
	In order for the maximum axial stress not to exceed the allowable stress, the area 
	$A$ of the truss member must satisfy 
	\begin{equation}
		\left| \frac{N_{max}}{A}\right| \leq \sigma_a
		\label{eqn:tol}
	\end{equation}
	where $N_{max}$ is the maximum axial force.
	For 
	\[ 
		N_{max}=-\frac{2F}{\sqrt{3}}, \ \ A=\frac{\pi}{4}d^2, 
	\]
	eq.(\ref{eqn:tol}) gives
	\begin{equation}
		d^2 \geq  \frac{8}{\sqrt{3}\pi} \frac{F}{\sigma_a}.
		\label{eqn:}
	\end{equation}
	Thus we obtain the minimum allowable diameter $d_{min}$ as 
	\begin{equation}
		d_{min}=
		\left\{ \frac{8}{\sqrt{3}\pi} \frac{F}{\sigma_a}
		\right\}^{1/2}.
		\label{eqn:dmin}
	\end{equation}
\item
	Using the given numerical data, $d_{min}$ is evaluated as shown in Fig.\ref{fig:fig4}.
	\begin{figure}[h]
	\begin{center}
	\includegraphics[width=1.0\linewidth]{fig2.eps} 
	\end{center}
	\caption{A table for the numerical evaluation of $d_{min}$ (the minimum allowable diameter of the truss member) .} 
	\label{fig:fig4}
	\end{figure}
\end{enumerate}
\end{document}
%%%%%%%%%%%%%%%%%%%%%%%%%%%%%%%%%%%%%%%%%%%%%%%%%%%%%%%%%%%%%%%%
