\documentclass[10pt,a4j]{article}
%\usepackage{graphicx,wrapfig}
\usepackage{graphicx}
\setlength{\topmargin}{-1.5cm}
%\setlength{\leftmargin}{1.5cm}
%\setlength{\textwidth}{15.5cm}
\setlength{\textheight}{25.2cm}
\newlength{\minitwocolumn}
\setlength{\minitwocolumn}{0.5\textwidth}
\addtolength{\minitwocolumn}{-\columnsep}
%\addtolength{\baselineskip}{-0.1\baselineskip}
%
\def\Mmaru#1{{\ooalign{\hfil#1\/\hfil\crcr
\raise.167ex\hbox{\mathhexbox 20D}}}}
%
\begin{document}
\newcommand{\fat}[1]{\mbox{\boldmath $#1$}}
\newcommand{\D}{\partial}
\newcommand{\w}{\omega}
\newcommand{\ga}{\alpha}
\newcommand{\gb}{\beta}
\newcommand{\gx}{\xi}
\newcommand{\gz}{\zeta}
\newcommand{\vhat}[1]{\hat{\fat{#1}}}
\newcommand{\spc}{\vspace{0.7\baselineskip}}
\newcommand{\halfspc}{\vspace{0.3\baselineskip}}
\bibliographystyle{unsrt}
%\pagestyle{empty}
\newcommand{\twofig}[2]
 {
   \begin{figure}
     \begin{minipage}[t]{\minitwocolumn}
         \begin{center}   #1
         \end{center}
     \end{minipage}
         \hspace{\columnsep}
     \begin{minipage}[t]{\minitwocolumn}
         \begin{center} #2
         \end{center}
     \end{minipage}
   \end{figure}
 }
%%%%%%%%%%%%%%%%%%%%%%%%%%%%%%%%%
%\vspace*{\baselineskip}
\begin{center}
	{\Large \bf Lecture Note 2, Civil Engineering I \\
	- introduction to structural mechanics-\\
	Equilibrium of Force and Moment
	}  \\
\end{center}
\vspace{1.5cm}
\setcounter{section}{1}
\section{Equilbrium of Force and Moment}
\subsection{Rotaional motion and moment}
Consider a perfectly rigid body $B$ subjected to a set of forces
\[ 
	\left\{\ \fat{f}_1, \fat{f}_2, \dots \fat{f}_{n} \right\}
\]
(see Fig.\ref{fig:fig3}-(a)).
If $B$ is staying still under the loading, the Newton's 2nd law requires that 
the total force vanishes. Thus,
\begin{equation}
	\sum_{i=1}^n \fat{f}_i = \fat{0}.
	\label{eqn:2nd_law}
\end{equation}
Note that eq.(\ref{eqn:2nd_law}) is a necessary and not a sufficient condition 
for the body to be in a static equilibrium. 
%A simple counter example for is shown in Fig.\ref{fig:fig3}-(b).
\begin{figure}[h]
	\begin{center}
	\includegraphics[width=0.6\linewidth]{fig3.eps} 
	\end{center}
	\caption{A body $B$ subjected to a set of forces $\fat{f}_i,(i=1\cdot n)$.} 
	\label{fig:fig3}
\end{figure}
To see this, consider a special loading condition shown in Fig.\ref{fig:fig3}. 
Our experience tells us that the body start rotating upon loading the couple force 
 shown in this figure although the sum $\fat{f}_1+\fat{f}_2$ of the forces is zero.
This means the Newton's 2nd law \ref{eqn:2nd_law} only ensures that the loads
do not cause a translational motion. 
It is therefore necessaray to introduce an additional condition concerning rotational 
motion to complete the equilibrium condition for a rigid body. 
To do so, we have to device a way to measure the tendency that a given force to have on 
the rotational motion. 
\subsection{Moment}
The tendency that a force to rotatate a body about given point, or an axis,
is called moment. To define the moment based on our experience, imagine that 
a body $B$ is pinned on a wall at $P$ about which $B$ can rotate freely without 
friction (see Fig.\ref{fig:fig1}). If a vertical load $\fat{f}$ is applied to a point $A$ 
located on the same level as $P$, the body starts rotating in ithe clockwise direction.
If we're asked to rotate the body twice as hard, we'd probably either 
double the magnitude of the force, or slide the point of application twice as far from $P$ 
as illustrated in Figs.\ref{eqn:fig1}-(b) and (c).
This observation suggests that the moment should be proportional to the magnitude of the 
force and the distance between the loading point and the pivot.  
Thus, we may adopt the following as our first tentative definition for the moment $M$. 
\begin{equation}
	M=({\rm distance \, between \,the \, pivot\, and\, force) \times (magnitude \, of \, the \,force)}
	\label{eqn:def_M_temp}
\end{equation}
According to the definition (\ref{eqn:def_M_temp}), the moments about $P$ due the 
forces shown in Figs.-(a), (b) and (c) are evaluated as $r|\fat{f}|, r|2\fat{f}|$ and $2r|\fat{f}|$, respectively.

The definition works fine as far as the configuration shown in shown in Fig.\ref{fig:fig1}.
To generalize the definition further, let's next inspect an obliquely applied forces
 shown in Fig.\ref{fig:fig2}.
In this setup, the force is not vertically directed, and the pivot and loading points are on the different levels.
If we decompose the force vector $\fat{f}$, as shown in Fig.\ref{fig:fig2} into the components parallell $\fat{f}_{\perp}$ 
and vertical $\fat{f}_\parallel$ to the line $l$ passing $P$ and $A$, we notice that $\fat{f}_{\parallel}$ does not 
contribute to the rotational motion about $P$. 
This means we need to consider only the vertical component $\fat{f}_{\perp}$ in defining the moment. 
Consequently, the present setup has been reduced to the previously inspected in Fig.\ref{fig:fig2}-(a).
Thus, we can now write the formula for the moment $M$ as 
\begin{equation}
	M=|\fat{r}|\times |\fat{f}_{\perp}|.
	\label{eqn:def_M_fp}
\end{equation}
Geometrically, eq.(\ref{eqn:def_M_fp}) gives the area $S$ of parallellogram $PABC$. 
The formula for $M$ is altanatively writen as
\begin{equation}
	M=r_\perp \times |\fat{f}|
	\label{eqn:}
\end{equation}
where $r_\perp$ is the parpendicular distance from $P$ to the line of loading 
$l'$ depicted in Fig.\ref{fig:fig2}.
Finally, we incorprate a mechanism to distinguish a clockwise from counterclockwise 
moment into the definition of $M$. This is achieved by assigining positive 
and negative moments to the anticlockwise and clockwise moments as follows.
\begin{equation}
	M=\sigma S , \ \ \left(S= |\fat{r}|\times |\fat{f}_{\perp}|=r_\perp \times |\fat{f}| \right)
	\label{eqn:M_def}
\end{equation}
with 
\begin{equation}
	\sigma=\left\{
		\begin{array}{cc}
			+1 & ({\rm counterclockwise}) \\
			-1 & ({\rm clockwise}) 
		\end{array}
	\right.
	,
	\label{eqn:sign}
\end{equation}
which completes our definition of moment. 
\begin{figure}[h]
	\begin{center}
	\includegraphics[width=1.0\linewidth]{fig1.eps} 
	\end{center}
	\caption{
		Basic loading conditions considered in defining moment that generate 
		rotational motion about a pivotal point $P$.} 
	\label{fig:fig1}
\end{figure}
%\begin{equation}
%	M=(0,0,xf_y-yf_x)
%	\label{eqn:}
%\end{equation}
\begin{figure}[h]
	\begin{center}
	\includegraphics[width=0.5\linewidth]{fig2.eps} 
	\end{center}
	\caption{Oblique load acting on a body $B$.} 
	\label{fig:fig2}
\end{figure}

In the previous class, we have learned the relationship between the vector 
cross product and the area of the parallelogram. In the present case, the 
it is written as 
\begin{equation}
	S=\left| \fat{r} \times \fat{f} \right|.
	\label{eqn:}
\end{equation}
Thus, we can define the moment using the vector cross product. 
In using vector cross product, however, $\fat{r}$ and $\fat{f}$ have 
to be considered as vectors in 3D space. 
The most straightforward way of embedding the 2D vectors in 3D space is 
to pad zero to the third coordinate writting 
\begin{equation}
	\fat{r}=(x,y,0), \ \ 
	\fat{f}=(f_x,f_y,0).
	\label{eqn:}
\end{equation}
Then the component-based formula of vector cross product yields the following. 
\begin{equation}
	\fat{r}\times \fat{f}=(0,0,xf_y-yf_x).
\end{equation}
We have thus obtained an alternative formula for $M$ as 
\begin{equation}
	M=\sigma \left| 
		xf_y-yf_x
	\right|.
\end{equation}
Furthermore, we can verify that 
\begin{equation}
	M=xf_y-yf_x.
\end{equation}
since $\sigma$, or the rotatinal direction, has already been considered in the cross product.
The definition (\ref{eqn:M_def}) of the moment is valid only for 2D problem where 
rotational motion take places about an axisi with  predetermined direction. 
When we have to deal with rotational motion in 3D space, moment is defined as a vector 
quantity by the cross product
\begin{equation}
	\fat{M}=(M_x, M_y, M_z)=
	\fat{r}\times\fat{f}
	=
	\left(
		yf_z-zf_y
	,
		zf_x-xf_z
	,
		xf_y-yf_x
	\right).
	\label{eqn:}
\end{equation}
This is not surprising because the axis of rotation can be directed in 
any direction in the three dimensional space depending on the loading condition. 
\subsection{Equilibrium Equations}
From the foregoing discussion, there should be two types of equilibrium conditions
for a rigid body. The one is the equilibrium of force and the equibrium of the 
moment is the other. Those are written as 
\begin{equation}
	\sum_{i=1}^n \fat{f}_i = \fat{0}, 
	\label{eqn:equib_f}
\end{equation}
and 
\begin{equation}
	\sum_{i=1}^n \fat{x}_i\times \fat{f}_i
	=
	\fat{0}
	\label{eqn:equib_M}
\end{equation}
where $\fat{x}_i$ is the loading point of $\fat{f}_i$.
It should be noted that $\fat{r}_i \times \fat{f}_i$ is the moment 
due to $\fat{f}_i$ about the origin of the coordinate. 
Equation (\ref{eqn:equib_M}) thus read 
"the sum of all moment about the coordinate origin is zero".
However, the total moment about an arbitrary point, say $\fat{p}$, 
should equally vanishes if the body is staying still. 
Thus, it seems necessary to write the equilbrium equation as 
\begin{equation}
	\sum_{i=1}^n \left(\fat{x}_i-\fat{p}\right) \times \fat{f}_i = \fat{0}
	\label{eqn:equib_Mp}
\end{equation}
instead of eq.(\ref{eqn:equib_M}).
However, eqs.(\ref{eqn:equib_f}) and (\ref{eqn:equib_M}) imply eq.(\ref{eqn:equib_Mp})
because 
\begin{equation}
	\sum_{i=1}^n \left(\fat{x}_i-\fat{p}\right) \times \fat{f}_i
	=
	\sum_{i=1}^n \left( \fat{x}_i\times \fat{f}_i  \right)
	- 
	\fat{p} \times \left(\sum_{i=1}^n \ \fat{f}_i \right), 
	\label{eqn:equib_Mp2}
\end{equation}
and by assumption the left hand side of eq.(\ref{eq:equib_Mp2}) vanish.
\section{Solved Problems}
\begin{enumerate}
\item
	Obtain the moments about $P_1,P_2,P_3$ and $P_4$ due to the vertical
	force $\fat{f}$ shown in Fig.\ref{fig:fig4}.
	\begin{figure}[h]
		\begin{center}
		\includegraphics[width=0.35\linewidth]{fig4.eps} 
		\end{center}
		\caption{A vertical force of maginitude $F$ acting downward to a body.} 
		\label{fig:fig4}
	\end{figure}
	\\

	{\small
		Let the pertpendicular distance from $A$ to $P_i$ be denoted as $h_i$ for $i=1,\dots 4$. 
		Then, $h_i$ are given as 
		\[
			h_1=2h, \ h_2=r, \ h_3=0, \ h_4=h.
		\]
		The moment $M_i$ about $P_i$ is given by 
		\[
			M_i=\pm \left| \fat{f} \right| h_i= \pm Fh_i
		\]	
		where the sign is determined so that the counterclockwise momemt has a positive value.
		Hence, the moments are given as
		\[
			M_1=-2Fr, \ M_2=-Fr, \ M_3=0, \ M_4=Fr.
		\]
	}
\item
	Obtain the moments about $P_1$ to $P_5$ due to the obliquely acting load $\fat{f}$ 
	of magnitude $F$ as shown in Fig.\ref{fig:fig6}.
	\begin{figure}[h]
		\begin{center}
		\includegraphics[width=0.35\linewidth]{fig6.eps} 
		\end{center}
		\caption{A force of magnitude $F$ acting obliquely to a body.} 
		\label{fig:fig6}
	\end{figure}
	\\
	{\small
		Let's denote the moment about $P_i$ as $M_i$ for $i=1,\dots 4$.
		Since the line of action of $\fat{f}$ passes $P_1$, we have $M_1=0$.
		If we note that $\overrightarrow{AP_2}$ is perpendicular to $\fat{f}$, 
		we obtain $M_2$ as 
		\[ 
			M_2=F\times \overline{AP}_2=\sqrt{2}Fr
		\].
		To obtain $M_3$ and $M_4$, it is convenient to decompose $\fat{f}$ into 
		the horizontal and vertical components as 
		\[
			\fat{f}=(f_x,f_y)=\frac{F}{\sqrt{2}}\left(1, -1\right).
		\]
		With $f_y$, $M_3$ and $M_4$ are evaluated as follows.
		\[
			M_3=+\left|f_y\right| \times \overline{A P_3}=\frac{Fr}{\sqrt{2}}
		\]
		\[
			M_4=+\left|f_y \right| \times \overline{A P_4}=\sqrt{2}Fr
		\]
		The moments are all positive since $\fat{f}$ generates the counterclockwise moments 
		with respect to $P_1$ to $P_4$.
	}
\item
	Determine the magnitude and the direction of the force $\fat{f}$ shown in 
	Fig.\ref{fig:fig8} necessary to equilibrate the four forces acting to the bar.
	Also, find if the total moment vanishes for the force thus determined.  
	\begin{figure}[h]
		\begin{center}
		\includegraphics[width=0.4\linewidth]{fig8.eps} 
		\end{center}
		\caption{A straight bar subjected to a set of forces.} 
		\label{fig:fig8}
	\end{figure}
	\\

	{\small
		Let $F$ be the unknwon magnitude of $\fat{f}$, and write $\fat{f}$ with 
		the vector component as 
		\[
			\fat{f}=F \left( -\cos\theta,\, \sin\theta \right)
		\]
		for the $xy$ Cartesian coordinate shown in Fig.\ref{fig:fig8}. 
		Then, the equiribrium equations of the horizontal and vertical forces are 
		given by 
		\[
			-F\cos\theta + 2[kgf]=0, \ \ 
			F\sin\theta -1[kgf]-1[kgf]=0.
		\]
		to find 
		\begin{equation}
			F=2\sqrt{2} \,[{\rm kgf}], \ \ \theta=\frac{\pi}{4}
			\label{eqn:ans3}
		\end{equation}
		For the force vector given by eq.(\ref{eqn:ans3}), we can show that the 
		the sum of the moments vanishes.
		For example, the total moment about $(x,y)=(0,0)$, that is the point where $\fat{f}$ 
		is applied, is evaluated as  
		\[
			{\rm 1[kgf] \times 2[m] -1[kgf] \times 2[m]=0 [kgf] }
		\]
		since the horizontal and the oblique force $\fat{f}$ do not contribute to 
		the moment about this point. We would conclude the same if 
		regardless of the point for which the moments are evaluated. 	
	}

\end{enumerate}
\section{Supplmentary Problems}
\begin{enumerate}
\item
	Obtain moments with respect to points $P_1,P_2,\dots P_4$ due to 
	the horizontal force $\fat{f}$ shown in Fig.\ref{fig:fig5}. 
	\begin{figure}[h]
		\begin{center}
		\includegraphics[width=0.35\linewidth]{fig5.eps} 
		\end{center}
		\caption{A horizontal force of magnitude $F$ acting to a body.} 
		\label{fig:fig5}
	\end{figure}
\item
	Obtain moments with respect to points $P_2,P_3,\dots P_5$
	due to the load of magnitude $F$ acting obliquely to 
	point $P_1$ as shown in Fig.\ref{fig:fig7}.
	\begin{figure}[h]
		\begin{center}
		\includegraphics[width=0.35\linewidth]{fig7.eps} 
		\end{center}
		\caption{A force of magnitude $F$ acting obliquely to a body.} 
		\label{fig:fig7}
	\end{figure}
\item
	Calculate the total force and the total moments about $P_1, P_2,\dots P_5$, 
	to show that the bar shown in Fig.\ref{fig:fig9} is in an equilibrium state 
	in terms of both the force and moment. 
	Note that $F$ and $F/2$ are the magnitude of the downward and upward forces 
	applied vertically to the bar, respectively.
	\begin{figure}[h]
		\begin{center}
		\includegraphics[width=0.4\linewidth]{fig9.eps} 
		\end{center}
		\caption{A straight bar subjected to a set of forces.} 
		\label{fig:fig9}
	\end{figure}
\end{enumerate}
\end{document}
%%%%%%%%%%%%%%%%%%%%%%%%%%%%%%%%%%%%%%%%%%%%%%%%%%%%%%%%%%%%%%%%
