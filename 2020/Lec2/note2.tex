\documentclass[10pt,a4j]{article}
%\usepackage{graphicx,wrapfig}
\usepackage{graphicx}
\setlength{\topmargin}{-1.5cm}
%\setlength{\leftmargin}{1.5cm}
%\setlength{\textwidth}{15.5cm}
\setlength{\textheight}{25.2cm}
\newlength{\minitwocolumn}
\setlength{\minitwocolumn}{0.5\textwidth}
\addtolength{\minitwocolumn}{-\columnsep}
%\addtolength{\baselineskip}{-0.1\baselineskip}
%
\def\Mmaru#1{{\ooalign{\hfil#1\/\hfil\crcr
\raise.167ex\hbox{\mathhexbox 20D}}}}
%
\begin{document}
\newcommand{\fat}[1]{\mbox{\boldmath $#1$}}
\newcommand{\D}{\partial}
\newcommand{\w}{\omega}
\newcommand{\ga}{\alpha}
\newcommand{\gb}{\beta}
\newcommand{\gx}{\xi}
\newcommand{\gz}{\zeta}
\newcommand{\vhat}[1]{\hat{\fat{#1}}}
\newcommand{\spc}{\vspace{0.7\baselineskip}}
\newcommand{\halfspc}{\vspace{0.3\baselineskip}}
\bibliographystyle{unsrt}
%\pagestyle{empty}
\newcommand{\twofig}[2]
 {
   \begin{figure}
     \begin{minipage}[t]{\minitwocolumn}
         \begin{center}   #1
         \end{center}
     \end{minipage}
         \hspace{\columnsep}
     \begin{minipage}[t]{\minitwocolumn}
         \begin{center} #2
         \end{center}
     \end{minipage}
   \end{figure}
 }
%%%%%%%%%%%%%%%%%%%%%%%%%%%%%%%%%
%\vspace*{\baselineskip}
\begin{center}
	{\Large \bf Lecture Note 2, Civil Engineering I \\
	- introduction to structural mechanics-\\
	Equilibrium of Force and Moment
	}  \\
\end{center}
\vspace{1.5cm}
\setcounter{section}{1}
\section{Equilbrium Condition}
Let's consider a perfectly rigid body $B$ is subjected to a set of forces:
\[ 
	\left\{\ \fat{f}_1, \fat{f}_2, \dots \fat{f}_{n} \right\}
\]
(see Fig.\ref{fig:fig3}-(a)).
If $B$ is staying still undre the loading, the sum of forces must vanishes
\begin{equation}
	\sum_{i=1}^n \fat{f}_i = \fat{0}
	\label{eqn:2nd_law}
\end{equation}
according to the Newton's 2nd law. This is a necessary but not a sufficient condition 
for the body to be in a static equilibrium. A counter example is shown in Fig.\ref{fig:fig3}-(b).
In this case, the body will start rotating although the total force $\fat{f}_1+\fat{f}_2$ vanishes.
The Newton's 2nd law \ref{eqn:2nd_law} ensures that the loading does not cause a 
 translational motion. 
 It is therefore necessaray to impose an addiional condition to prevent the body from rotating. 
 To do so, we first need to qunatify the tendency that a given force to cause a rotational motion. 
The quantitative measure of such tendency is called moment, which is introduced as follows. 

The moment should be proportinal to both the distance and the maginitude of the force. 
\begin{equation}
	M=|\fat{r}| \times |\fat{f}_{\perp}|=S({\rm area \  of \ PABC})
	\label{eqn:}
\end{equation}
\begin{equation}
	M=r_\perp \times |\fat{f}|
	\label{eqn:}
\end{equation}
\begin{equation}
	M=S \times \left\{
		\begin{array}{cc}
			+1 & (counterclockwise) \\
			-1 & (clockwise) 
		\end{array}
	\right.
	\label{eqn:}
\end{equation}
\begin{equation}
	M=\fat{r} \times \fat{f}
	\label{eqn:}
\end{equation}
\begin{equation}
	\fat{r}=(x,y,0), \ \ 
	\fat{f}=(f_x,f_y,0)
	\label{eqn:}
\end{equation}
\begin{equation}
	M=(0,0,xf_y-yf_x)
	\label{eqn:}
\end{equation}
\begin{equation}
	\sum_{i=1}^n \fat{x}_i\times \fat{f}_i
	=
	\fat{0}
	\label{eqn:}
\end{equation}
\begin{equation}
	\sum_{i=1}^n \left(\fat{x}_i-\fat{p}\right) \times \fat{f}_i
	=
	\sum_{i=1}^n \left( \fat{x}_i\times \fat{f}_i  \right)
	- 
	\fat{p} \times \left(\sum_{i=1}^n \ \fat{f}_i \right)
	=\fat{0}
	\label{eqn:}
\end{equation}
\begin{figure}[h]
	\begin{center}
	\includegraphics[width=0.6\linewidth]{fig3.eps} 
	\end{center}
	\caption{A body $B$ subjected to a set of forces $\fat{f}_i,(i=1\cdot n)$.} 
	\label{fig:fig3}
\end{figure}
\begin{figure}[h]
	\begin{center}
	\includegraphics[width=1.0\linewidth]{fig1.eps} 
	\end{center}
	\caption{Three equivalent forces in terms of the moment about pivot $P$.} 
	\label{fig:fig1}
\end{figure}
\begin{figure}[h]
	\begin{center}
	\includegraphics[width=0.5\linewidth]{fig2.eps} 
	\end{center}
	\caption{Oblique load acting on a body $B$.} 
	\label{fig:fig2}
\end{figure}
\section*{Excersises 2}
\begin{itemize}
\item
Exercise 2-1:\\
Obtain moments with respect to points $P_1,P_2,\dots P_4$ due to the downward 
vertical force $\fat{f}$ shown in Fig.\ref{fig:fig2_1} -(a).
\item
Exercise 2-2:\\
Obtain moments with respect to points $P_1,P_2,\dots P_4$ due to 
the horizontal force $\fat{f}$ shown in Fig.\ref{fig:fig2_1}-(b). 
\item
Exercise 2-3:\\
Obtain moments with respect to points $P_2,P_3,\dots P_5$
due to the load of magnitude $F$ acting obliquely to point $P_1$ 
as shown in Fig.\ref{fig:fig2_2}-(a).
\item
Exercise 2-4:\\
Obtain moments with respect to points $P_1,P_2,\dots P_4$
 due to the force $\fat{f}$ acting obliquely to point $P_5$ as 
		shown in Fig.\ref{fig:fig2_2}-(b).
\item
Exercise 2-5:\\
	Determine the magnitude of the force $\fat{f}$ shown in 
	Fig.\ref{fig:fig2_3}-(a) that is necessary to equilibrate 
	the bar both in terms of the force and moment. 
\item
Exercise 2-6:\\
	Calculate the total force and the total moments about $P_1, P_2,\dots P_5$, 
	to show that the bar shown in Fig.\ref{fig:fig2_3}-(b) is in an equilibrium state 
	in terms of both the force and moment. 
	Note that $F$ and $F/2$ are the magnitude of the downward and upward forces 
	applied vertically to the bar, respectively.
\end{itemize}
\newpage
\begin{figure}[h]
	\begin{center}
	\includegraphics[width=0.7\linewidth]{fig2_1.eps} 
	\end{center}
	\caption{Vertical and horizontal forces acting on a body.} 
	\label{fig:fig2_1}
\end{figure}
\begin{figure}[h]
	\begin{center}
	\includegraphics[width=0.7\linewidth]{fig2_2.eps} 
	\end{center}
	\caption{Forces acting obliquely to a body.} 
	\label{fig:fig2_2}
\end{figure}
\begin{figure}[h]
	\begin{center}
	\includegraphics[width=0.8\linewidth]{fig2_3.eps} 
	\end{center}
	\caption{Straight bars subjected to a set of forces.} 
	\label{fig:fig2_3}
\end{figure}
\end{document}
%%%%%%%%%%%%%%%%%%%%%%%%%%%%%%%%%%%%%%%%%%%%%%%%%%%%%%%%%%%%%%%%
