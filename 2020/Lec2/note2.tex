\documentclass[10pt,a4j]{article}
%\usepackage{graphicx,wrapfig}
\usepackage{graphicx}
%\usepackage{showkeys}
\setlength{\topmargin}{-1.5cm}
%\setlength{\leftmargin}{1.5cm}
%\setlength{\textwidth}{15.5cm}
\setlength{\textheight}{25.2cm}
\newlength{\minitwocolumn}
\setlength{\minitwocolumn}{0.5\textwidth}
\addtolength{\minitwocolumn}{-\columnsep}
%\addtolength{\baselineskip}{-0.1\baselineskip}
%
\def\Mmaru#1{{\ooalign{\hfil#1\/\hfil\crcr
\raise.167ex\hbox{\mathhexbox 20D}}}}
%
\begin{document}
\newcommand{\fat}[1]{\mbox{\boldmath $#1$}}
\newcommand{\D}{\partial}
\newcommand{\w}{\omega}
\newcommand{\ga}{\alpha}
\newcommand{\gb}{\beta}
\newcommand{\gx}{\xi}
\newcommand{\gz}{\zeta}
\newcommand{\vhat}[1]{\hat{\fat{#1}}}
\newcommand{\spc}{\vspace{0.7\baselineskip}}
\newcommand{\halfspc}{\vspace{0.3\baselineskip}}
\bibliographystyle{unsrt}
%\pagestyle{empty}
\newcommand{\twofig}[2]
 {
   \begin{figure}
     \begin{minipage}[t]{\minitwocolumn}
         \begin{center}   #1
         \end{center}
     \end{minipage}
         \hspace{\columnsep}
     \begin{minipage}[t]{\minitwocolumn}
         \begin{center} #2
         \end{center}
     \end{minipage}
   \end{figure}
 }
%%%%%%%%%%%%%%%%%%%%%%%%%%%%%%%%%
%\vspace*{\baselineskip}
\begin{flushright}
	Civil Engineering I\\
	April 30, 2020
\end{flushright}
\begin{center}
	{\LARGE \bf Lecture Note 2}
\end{center}
\setcounter{section}{1}
\section{Equilibrium of Force and Moment}
\subsection{Rotational motion and moment}
Consider a rigid body $B$ subjected to a set of forces
\[ 
	\left\{\ \fat{f}_1, \fat{f}_2, \dots \fat{f}_{n} \right\}
\]
(see Fig.\ref{fig:fig3}-(a)).
If body $B$ is staying perfectly still, 
Newton's second law requires the sum of the forces to vanish.
Thus,
\begin{equation}
	\sum_{i=1}^n \fat{f}_i = \fat{0}.
	\label{eqn:2nd_law}
\end{equation}
Note that the converse is not necessarily true.
To see this, consider a special loading condition depicted in Fig.\ref{fig:fig3}-(b). 
\begin{figure}[h]
	\begin{center}
	\includegraphics[width=0.6\linewidth]{fig3.eps} 
	\end{center}
	\caption{A body $B$ subjected to a set of forces $\fat{f}_i,(i=1\cdot n)$.} 
	\label{fig:fig3}
\end{figure}
Although the couple-force satisfies the second law of eq.(\ref{eqn:2nd_law}), we will see the body starts rotating. 
This means Newton's 2nd law ensures that the vanishing forces do not generate translational 
motion but not necessarily rotation. It is, therefore, necessary to impose an additional 
condition concerning rotation to complete the equilibrium condition. 
To do so, we first need to devise a way to measure the tendency of a force to rotate a body.
\subsection{Moment}
The tendency of a force to rotate a body about a given point, or an axis, is
called moment. To define the moment based on our experience, 
imagine that body $B$ is pinned on a wall at $P$ about which $B$ can rotate freely without 
friction (see Fig.\ref{fig:fig1}). When a load $\fat{f}$ is applied vertically to a point $A$ 
located on the same level as $P$, the body starts rotating in the clockwise direction. 
 If we are asked to rotate the body twice as hard, we would either double the magnitude of 
the force or slide the point of application twice as far from $P$ as illustrated in Figs.\ref{eqn:fig1}-(b) and (c). 
The above observations suggest that the moment should be proportional to the magnitude of the force 
and the distance between the pivot and loading point. 
Thus, we adopt the following as our tentative definition of moment $M$.
\begin{equation}
	M=({\rm distance \, between \,the \, pivot\, and\, loading\ point) \times (magnitude \, of \, the \,force)}
	\label{eqn:def_M_temp}
\end{equation}
According to the definition of (\ref{eqn:def_M_temp}), the moments about $P$ due to the 
forces shown in Fig.\ref{fig:fig1}-(a), (b) and (c) are evaluated as $r|\fat{f}|, r|2\fat{f}|$ 
and $2r|\fat{f}|$, respectively. The definition works fine as far as the pivot-force 
configuration shown in Fig.\ref{fig:fig1} is concerned.
To extend the definition to more general configuration, let's next inspect the setup 
shown in Fig.\ref{fig:fig2}. In this setup, an angled force is applied to a point $A$ 
located on a level different from the pivot $P$.
To reduce the general configuration in Fig.\ref{fig:fig2} to the special case 
in Fig.\ref{fig:fig1}-(a), we decompose the force vector $\fat{f}$ into the parallel 
$\fat{f}_{\perp}$ and vertical $\fat{f}_\parallel$ components with respect to the line $l$ 
connecting $P$ and $A$. We then notice that $\fat{f}_{\parallel}$ does not contribute to 
the rotation about $P$. This means we need to retain only the vertical component 
$\fat{f}_{\perp}$ in defining moment $M$. Thus, we can formulate the definition of (\ref{eqn:def_M_temp}) 
for general pivot-force configuration as 
\begin{equation}
	M=|\fat{r}|\times |\fat{f}_{\perp}|.
	\label{eqn:def_M_fp}
\end{equation}
Geometrically, the right hand side of eq.(\ref{eqn:def_M_fp}) gives the area $S$ of 
parallelogram $PABC$. Therefore, the formula for $M$ can be written alternatively as
\begin{equation}
	M=r_\perp \times |\fat{f}|
	\label{eqn:}
\end{equation}
where $r_\perp$ is the perpendicular distance from $P$ to the line of loading 
$l'$ shown in Fig.\ref{fig:fig2}.
Finally, we incorporate a mechanism to distinguish a clockwise from counterclockwise 
moment into the definition of $M$. This is accomplished by assigning negative or  
positive moment as follows.
\begin{equation}
	M=\sigma S , \ \ \left(S= |\fat{r}|\times |\fat{f}_{\perp}|=r_\perp \times |\fat{f}| \right)
	\label{eqn:M_def}
\end{equation}
with 
\begin{equation}
	\sigma=\left\{
		\begin{array}{cc}
			+1 & ({\rm counterclockwise}) \\
			-1 & ({\rm clockwise}) 
		\end{array}
	\right.
	,
	\label{eqn:sign}
\end{equation}
Equation (\ref{eqn:M_def}) with (\ref{eqn:sign}) is a complete definition of moment in two dimensional space. 
\begin{figure}[h]
	\begin{center}
	\includegraphics[width=1.0\linewidth]{fig1.eps} 
	\end{center}
	\caption{
		Basic loading conditions considered in defining moment with 
		respect to the pivotal point $P$.} 
	\label{fig:fig1}
\end{figure}
\begin{figure}[h]
	\begin{center}
	\includegraphics[width=0.5\linewidth]{fig2.eps} 
	\end{center}
	\caption{The configuration of the angled force, loading point and pivot as assumed rotation center.} 
	\label{fig:fig2}
\end{figure}
\subsection{Moment and Vector Cross Product}
In the previous lecture, we learned the relationship between the vector 
cross product and the area of a parallelogram. For parallelogram PABC in 
Fig.\ref{fig:fig2}, the relationship is  written as 
\begin{equation}
	S=\left| \fat{r} \times \fat{f} \right|.
	\label{eqn:r_x_f}
\end{equation}
With eq.(\ref{eqn:r_x_f}), the definition of $M$ in eq.(\ref{eqn:M_def}) can be rewritten as 
\begin{equation}
	M=\sigma\left| \fat{r} \times \fat{f} \right|
	\label{eqn:M_def2}
\end{equation}
In rewriting the definition, $\fat{r}$ and $\fat{f}$ have to be considered as vectors 
in 3D space. The most straightforward way of embedding 2D vectors in 3D space is 
to pad the third coordinate with zero as 
\begin{equation}
	\fat{r}=(x,\, y,\, 0), \ \ 
	\fat{f}=(f_x,\, f_y,\,0).
	\label{eqn:}
\end{equation}
Then the component-based formula of vector cross product yields 
\begin{equation}
	\fat{r}\times \fat{f}=(0,0,xf_y-yf_x)
\end{equation}
to have an alternative formula for $M$ as 
\begin{equation}
	M=\sigma \left| xf_y-yf_x \right| = xf_y-yf_x.
	\label{eqn:M_def2d_cross}
\end{equation}
The last equality in eq.(\ref{eqn:M_def2d_cross}) holds because 
$\fat{r}, \fat{f}$ and $\fat{r}\times\fat{f}$ form a right-handed system.
When $\fat{f}$ tends to generate a counterclockwise rotation, 
$\fat{r}\times\fat{f}$ directs upward, which means $xf_y-y_fx>0$.  
On the other hand, $\fat{r}\times\fat{f}$ directs downward when 
$\fat{f}$ tends to generate a clockwise rotation. \\

The definition (\ref{eqn:M_def}) and the alternative (\ref{eqn:M_def2d_cross}) 
are valid only for 2D problem where rotation takes place about the 3rd axis. 
When we deal with rotational motion in 3D space, moment is defined as a vector 
quantity as 
\begin{eqnarray}
	\fat{M} &=& 
	(M_x,\, M_y,\, M_z) \nonumber \\
	&=&
	\fat{r}\times\fat{f} \nonumber \\
	&=&
	\left(
		yf_z-zf_y
	,\,
		zf_x-xf_z
	,\,
		xf_y-yf_x
	\right).
	\label{eqn:}
\end{eqnarray}
This is not surprising because we can consider rotation about 
three mutually orthogonal axis in 3D space that there should be 
three corresponding moments.
\subsection{Equilibrium Equations}
Going back to the general loading condition illustrated in Fig.\ref{fig:fig3}-(a), 
we can now write two kinds of equilibrium conditions 
for a rigid body as 
\begin{equation}
	\sum_{i=1}^n \fat{f}_i = \fat{0}, 
	\label{eqn:equib_f}
\end{equation}
and 
\begin{equation}
	\sum_{i=1}^n \fat{x}_i\times \fat{f}_i
	=
	\fat{0}
	\label{eqn:equib_M}
\end{equation}
where $\fat{x}_i$ is the loading point of $\fat{f}_i$.
Equation (\ref{eqn:equib_f}) is the direct application of the 2nd law, 
while eq.(\ref{eqn:equib_M}) is the equilibrium condition for moment.
Since $\fat{r}_i \times \fat{f}_i$ is the moment due to $\fat{f}_i$ about the 
coordinate origin, eq.(\ref{eqn:equib_M}) read "the sum of all moment about 
the coordinate origin is zero".
Equations (\ref{eqn:equib_f}) and (\ref{eqn:equib_M}) are the complete
set of equilibrium conditions for a rigid body to stay perfectly still.
If both eqs.(\ref{eqn:equib_f}) and (\ref{eqn:equib_M}) are satisfied, 
the total moment vanishes regardless of the location of the assumed 
rotation center. To prove this, let $\fat{p}$ be the rotation center and 
examine the sum of moments about  $\fat{p}$. Then we have 
\begin{equation}
	\sum_{i=1}^n \left(\fat{x}_i-\fat{p}\right) \times \fat{f}_i
	=
	\sum_{i=1}^n \left( \fat{x}_i\times \fat{f}_i  \right)
	- 
	\fat{p} \times \left(\sum_{i=1}^n \ \fat{f}_i \right). 
	\label{eqn:equib_Mp2}
\end{equation}
It follows from eqs.(\ref{eqn:equib_f})-(\ref{eqn:equib_Mp2}) that 
\begin{equation}
	\sum_{i=1}^n \left(\fat{x}_i-\fat{p}\right) \times \fat{f}_i = \fat{0}
	\label{eqn:equib_Mp}
\end{equation}
holds for arbitrary $\fat{p}$.
\section{Solved Problems}
\begin{enumerate}
\item
	Obtain the moment about $P_1,P_2,P_3$ and $P_4$ due to the vertical
	force $\fat{f}$ shown in Fig.\ref{fig:fig4}.
	\begin{figure}[h]
		\begin{center}
		\includegraphics[width=0.35\linewidth]{fig4.eps} 
		\end{center}
		\caption{A vertical force of magnitude $F$ applied to the point $A$ in 
		a 2D space.} 
		\label{fig:fig4}
	\end{figure}
	\\

	{\small
		Let the perpendicular distance from $A$ to $P_i$ be $h_i$ for $i=1,\dots 4$(see Fig.\ref{fig:fig4_1}). 
		Then, $h_i$ are given as 
		\[
			h_1=2r, \ h_2=r, \ h_3=0, \ h_4=r.
		\]
		With the perpendicular distances, the moment $M_i$ about $P_i$ is written as 
		\[
			M_i=\pm \left| \fat{f} \right| h_i= \pm Fh_i
		\]	
		where the sign is chosen so that a counterclockwise moment takes positive value.
		Based on this rule, we have 
		\[
			M_1=-2Fr, \ M_2=-Fr, \ M_3=0, \ M_4=Fr.
		\]
		\begin{figure}[h]
		\begin{center}
		\includegraphics[width=0.35\linewidth]{fig4_1.eps} 
		\end{center}
			\caption{The vertical distances $h_1,h_2$ and $h_4$. $h_3$ is not 
			indicated in this figure as $h_3=0$.}
		\label{fig:fig4_1}
		\end{figure}
	}
\item
	Obtain the moments about $P_1$ to $P_4$ due to the obliquely acting load $\fat{f}$ 
	of magnitude $F$ as shown in Fig.\ref{fig:fig6}.
	\begin{figure}[h]
		\begin{center}
		\includegraphics[width=0.35\linewidth]{fig6.eps} 
		\end{center}
		\caption{A force of magnitude $F$ acting obliquely to the point $A$ in a 2D space.} 
		\label{fig:fig6}
	\end{figure}
	\\
	{\small
		Let the moment about $P_i$ be denoted as $M_i$ for $i=1,\dots 4$.
		By extending the line of action of $\fat{f}$ as indicated by 
		a red line in Fig.\ref{fig:fig6_1}, we immediately have $M_1=0$.
		Noting that $\overrightarrow{AP_2}$ is perpendicular to $\fat{f}$, 
		we have 
		\[ 
			M_2=F\times \overline{AP}_2=\sqrt{2}Fr.
		\]
		To obtain $M_3$ and $M_4$, it is convenient to decompose $\fat{f}$ into 
		the horizontal and vertical components as 
		\[
			\fat{f}=(f_x,f_y)=\frac{F}{\sqrt{2}}\left(1, -1\right).
		\]
		With $f_y$, $M_3$ and $M_4$ are given as 
		\[
			M_3=+\left|f_y\right| \times \overline{A P_3}=\frac{Fr}{\sqrt{2}}
		\]
		\[
			M_4=+\left|f_y \right| \times \overline{A P_4}=\sqrt{2}Fr.
		\]
		The moments above take positive values since $\fat{f}$ contributes 
		to the counterclockwise rotation about $P_2, P_3$ and $P_4$.
		\begin{figure}[h]
		\begin{center}
		\includegraphics[width=0.35\linewidth]{fig6_1.eps} 
		\end{center}
			\caption{The line of action and the vector components of $\fat{f}$.} 
		\label{fig:fig6_1}
		\end{figure}
	}
\item
	Determine the magnitude and direction of the force $\fat{f}$ shown in 
	Fig.\ref{fig:fig8} necessary to equilibrate the forces acting to the bar.
	For the force $\fat{f}$ thus determined, find if the total moment vanishes as well.
	\begin{figure}[h]
		\begin{center}
		\includegraphics[width=0.4\linewidth]{fig8.eps} 
		\end{center}
		\caption{A straight bar subjected to a set of forces.} 
		\label{fig:fig8}
	\end{figure}
	\\

	{\small
		Let $F$ be the magnitude of $\fat{f}$ and write $\fat{f}$ as 
		\[
			\fat{f}=F \left( -\cos\theta,\, \sin\theta \right)
		\]
		in terms of the $xy$ Cartesian vector components. 
		Then, the equilibrium equations of the horizontal and vertical 
		forces are written as 
		\[
			-F\cos\theta + 2[{\rm kgf}]=0, \ \ 
			F\sin\theta -1[{\rm kgf}]-1[{\rm kgf}]=0
		\]
		to have 
		\begin{equation}
			F=2\sqrt{2} \,[{\rm kgf}], \ \ \theta=\frac{\pi}{4}.
			\label{eqn:ans3}
		\end{equation}
		For the force vector given by eq.(\ref{eqn:ans3}), we can show that the 
		the sum of the moments vanishes.
		For example, the total moment about $(x,y)=(0,0)$ is shown to vanish as  
		\[
			{\rm 1[kgf] \times 2[m] -1[kgf] \times 2[m]=0 [kgf] }
		\]
		since the horizontal and the oblique force $\fat{f}$ do not 
		contribute to the moment about $(0,\, 0)$. 
		We would conclude the same even if we evaluate the total moment 
		about an arbitrary off-centered point.
	}

\end{enumerate}
\section{Supplementary Problems}
\begin{enumerate}
\item
	Obtain the moments with respect to $P_1,\dots P_4$ shown 
	in Fig.\ref{fig:fig5} due to the horizontal force $\fat{f}$.  
	\begin{figure}[h]
		\begin{center}
		\includegraphics[width=0.35\linewidth]{fig5.eps} 
		\end{center}
		\caption{A horizontal force of magnitude $F$ acting to the point $A$ 
		in a 2D space.} 
		\label{fig:fig5}
	\end{figure}
\item
	Obtain the moments with respect to $P_1,\dots P_4$ shown in Fig.\ref{fig:fig7} 
	due to the force $\fat{f}$ of magnitude $F$ acting obliquely to 
	a point $A$ in a 2D space.
	\begin{figure}[h]
		\begin{center}
		\includegraphics[width=0.35\linewidth]{fig7.eps} 
		\end{center}
		\caption{An angled  force of magnitude $F$.} 
		\label{fig:fig7}
	\end{figure}
\item
	Obtain the total moments about $P_1,\dots P_4$ due to the 
	forces shown in Fig.\ref{fig:fig9}. Show that the total moment about 
	$P_5$ vanishes for arbitrary $a$ such that $0\leq a \leq l$.
	\begin{figure}[h]
		\begin{center}
		\includegraphics[width=0.4\linewidth]{fig9.eps} 
		\end{center}
		\caption{A straight bar subjected to a set of forces.} 
		\label{fig:fig9}
	\end{figure}
\end{enumerate}
\end{document}
%%%%%%%%%%%%%%%%%%%%%%%%%%%%%%%%%%%%%%%%%%%%%%%%%%%%%%%%%%%%%%%%
