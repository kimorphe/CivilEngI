\documentclass[10pt,a4j]{article}
%\usepackage{graphicx,wrapfig}
\usepackage{graphicx}
\setlength{\topmargin}{-1.5cm}
\setlength{\textwidth}{16.5cm}
\setlength{\textheight}{25.2cm}
\newlength{\minitwocolumn}
\setlength{\minitwocolumn}{0.5\textwidth}
\addtolength{\minitwocolumn}{-\columnsep}
%\addtolength{\baselineskip}{-0.1\baselineskip}
%
\def\Mmaru#1{{\ooalign{\hfil#1\/\hfil\crcr
\raise.167ex\hbox{\mathhexbox 20D}}}}
%
\begin{document}
\newcommand{\fat}[1]{\mbox{\boldmath $#1$}}
\newcommand{\D}{\partial}
\newcommand{\w}{\omega}
\newcommand{\ga}{\alpha}
\newcommand{\gb}{\beta}
\newcommand{\gx}{\xi}
\newcommand{\gz}{\zeta}
\newcommand{\vhat}[1]{\hat{\fat{#1}}}
\newcommand{\spc}{\vspace{0.7\baselineskip}}
\newcommand{\halfspc}{\vspace{0.3\baselineskip}}
\bibliographystyle{unsrt}
%\pagestyle{empty}
\newcommand{\twofig}[2]
 {
   \begin{figure}
     \begin{minipage}[t]{\minitwocolumn}
         \begin{center}   #1
         \end{center}
     \end{minipage}
         \hspace{\columnsep}
     \begin{minipage}[t]{\minitwocolumn}
         \begin{center} #2
         \end{center}
     \end{minipage}
   \end{figure}
 }
%%%%%%%%%%%%%%%%%%%%%%%%%%%%%%%%%
%\vspace*{\baselineskip}
\begin{center}
	{\Large \bf Lecture Note 1, Civil Engineering I \\ - an introduction to structural mechanics}  
\end{center}
%%%%%%%%%%%%%%%%%%%%%%%%%%%%%%%%%%%%%%%%%%%%%%%%%%%%%%%%%%%%%%%%
\section{Review of Vector Algebra}
\subsection{Scalar and vector quantites}
A scalar is a quantitiy that has only a magnitude, while a vector has 
both magnitude(or length) and direction.
%The magnitude can either be positive or negative. 
The examples of scalar quantities frequently used in physics and engineering 
are mass, temperature, concentration, length, weight and so on.
%Vector is a quantity that has both a magnitude and direction. 
The examples of vector quantities are displacement, velocity, acceleration, 
force, heat flux, mass flux, electric poloarization, etc. 
It is sometimes useful to denote vectors by arrows typically when 
the gemterical relationship among the vectors is of interest.
In that situation, the magnitude and direction of a vector 
are indicated by the length and the orientation of the arrow, 
respectively, as shown in Fig.\ref{fig:fig1_1}. 
For more rigorous discussion, however, symbolic representations 
of vectors are far more convenient. 
\begin{figure}[h]
	\begin{center}
	\includegraphics[width=0.2\linewidth]{arrow.eps} 
	\end{center}
	\caption{A vector $\fat{a}$ shown graphically by an arrow.} 
	\label{fig:fig1_1}
\end{figure}
To denote vector quantities in typed manuscripts symbolically, boldface letters such 
as $\fat{x},\fat{u},\fat{v},\dots$ are used instead of writting simply as $x, u$ and $v$.
It is important not to confuse vectors and scalars since they are qualitatively different entities.
To denote the magnitude of vector $\fat{a}$, we either write $|\fat{a}|$ or simply $a$.
Unlike scalars, vector length is always greater than or equal to zero(i.e. $|\fat{a}| \geq 0$).
If $\fat{a}$ has a unit magnitude ($|\fat{a}|=a=1$), then $\fat{a}$ is said to be a unit vector. 
If $|\fat{a}|=0$, it is said to be zero vector writting $\fat{a}=\fat{0}$. 
Note that the direction is not defined for zero vector.
\subsection{Vector Identity}
Two vectors $\fat{a}$ and $\fat{b}$ are considered identical 
if their magnitude and direction are equal.
In that case, we write $\fat{a}=\fat{b}$ regardless of their locations(see, Fig.\ref{fig:fig1_2}-(a)). 
For example, when two forces ($\fat{f}, \fat{g}$) of equal magnitude and direction 
are acting to two different locations on a body, 
$\fat{f}$ and $\fat{g}$ are considered mathematically identical although they 
are physically different forces(Fig.\ref{fig:fig1_2}-(b)).
The exception of this rule is position vectors. 
The origin of position vectors are always taken at 
the origin of a predefined coordinate as illustrated in Fig.\ref{fig:fig1_2}-(c).
The position vectors should be defined in that way since the correspondance between 
spatial points and position vectors must be one-to-one. 
\begin{figure}[h]
	\begin{center}
	\includegraphics[width=0.7\linewidth]{identity.eps} 
	\end{center}
	\caption{(a),(b)Pairs of mathematically identical vectors. 
	(c) An illustration of position vectors.} 
	\label{fig:fig1_2}
\end{figure}
\subsection{Basic Algebraic Operation on Vectors}
The most fundamental algebraic operations on vectors are (1)scalar mutiplication 
and (2)vector additions, which may be introduced as follows.
\subsubsection{Scalar multiplication}
A multiplication between a scalar $s$ and a vector $a$ are called 
"scalar multiplication" and is written as $s\fat{a}$. 
For $s>0$, scalar multiplication $s\fat{a}$ streches(compresses) $\fat{a}$ by  $s$ times 
if $s \geq 1(s<1)$ without changing the direction of $\fat{a}$.
When $s<0$, $s\fat{a}$ stretches or compresses $\fat{a}$ by $|s|$ times after reversing 
the direction of $\fat{a}$. 
Two obvious consequences of this definition of scalar multiplication are:
\begin{itemize}
\item 
	$1\fat{a}=\fat{a}$
\item 
	$0\fat{a}=\fat{0}$
\end{itemize}
\begin{figure}[h]
	\begin{center}
	\includegraphics[width=0.7\linewidth]{smul.eps} 
	\end{center}
	\caption{The results of scalar multiplications $s\fat{a}$ dependent on the scalar $s$.} 
	\label{fig:fig1_3}
\end{figure}
\subsubsection{Vector addition}
Vector addition is an operation that produces a new vector by connecting a pair of vectors. 
If we let $\fat{a}$ and $\fat{b}$ be two arbitrary vectors, 
we can form a directed broken line by connecting the tail of one vector with 
the head of the other. The sum $\fat{a}+\fat{b}$ is defined to be a vector 
extending from the tail to the head of the broken line.
The direct consenquece of this definition is 
\begin{equation}
	\fat{a}+\fat{0} = \fat{0}+\fat{a}=\fat{a}. 
\end{equation}
Note that vector addition is "commutative" since 
\[
	\fat{a}+\fat{b}=\fat{b}+\fat{a}
\]
alwyas hold true. This means the order of summation may is immaterial and can be 
exchanged. Clearly, we can add more than two vectors by repeating the pair wise vectors addition.
For example, we can perform summation of three vectors either by 
\[
	\fat{a}+\left( \fat{b}+\fat{c} \right)
\]
or by 
\[
	\left( \fat{a}+\fat{b} \right) + \fat{c}.
\]
We can prove graphically that the addition by result in a same vector. Hence, 
\begin{equation}
	\fat{a}+\left( \fat{b}+\fat{c} \right)
	=
	\left( \fat{a}+\fat{b} \right) + \fat{c}.
\end{equation}
\begin{figure}[h]
	\begin{center}
	\includegraphics[width=0.6\linewidth]{vadd.eps} 
	\end{center}
	\caption{(a) Vector addition $\fat{a}+\fat{b}$ and 
	(b) Vector subtraction $\fat{a}-\fat{b}$.} 
	\label{fig:1_4}
\end{figure}
\subsection{Inner (or Dot) product} 
\[
	\fat{a}\cdot\fat{b}=|\fat{a}||\fat{b}|\cos\theta
\]
\subsection{Exterior (or Cross) product} 
When a pair of vectors $\fat{a}$ and $\fat{b}$ make a right 
angle, they are said to be mutually orothogonal.  
\[
	\fat{a}\times \fat{b}= \fat{c}
	\Rightarrow
	\fat{a}\cdot\fat{c}=\fat{b}\cdot\fat{c}=0
	\left|c\right|^2=|\fat{a}|^2|\fat{b}|^2-\left(\fat{a}\cdot\fat{b}\right)^2
\]
\section{Newton's law of motion}
Newton's law of motion may be stated as follows.
\begin{itemize}
\item 
	There exsist a coordinate system in which a body moves at a constant velocity as long as the 
	no force is exserted on it. Such coordinate system is called an "inertia system".
		(exsistence of an intertia system)
\item
	The time rate of change in the linear momentum of a body is equal to the force applied to the body. 
	This relation may be formulated as 
	\[
		\fat{f}=\frac{d}{dt}m\fat{v}
	\]
		where $m, \fat{f}$ and $\fat{v}$ denote the mass, force and velocity, respectively. 
	(convervation of linear momentum, or equation of motion)
\item
	If a body A is applying a force to the other and B are internacting, 
	The law of action and reaction.
\end{itemize}
\end{document}
