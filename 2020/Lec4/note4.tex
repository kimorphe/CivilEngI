\documentclass[10pt,a4j]{article}
%\usepackage{graphicx,wrapfig}
\usepackage{graphicx}
\usepackage{showkeys}
\setlength{\topmargin}{-1.5cm}
%\setlength{\leftmargin}{1.5cm}
%\setlength{\textwidth}{15.5cm}
\setlength{\textheight}{25.2cm}
\newlength{\minitwocolumn}
\setlength{\minitwocolumn}{0.5\textwidth}
\addtolength{\minitwocolumn}{-\columnsep}
%\addtolength{\baselineskip}{-0.1\baselineskip}
%
\def\Mmaru#1{{\ooalign{\hfil#1\/\hfil\crcr
\raise.167ex\hbox{\mathhexbox 20D}}}}
%
\begin{document}
\newcommand{\fat}[1]{\mbox{\boldmath $#1$}}
\newcommand{\D}{\partial}
\newcommand{\w}{\omega}
\newcommand{\ga}{\alpha}
\newcommand{\gb}{\beta}
\newcommand{\gx}{\xi}
\newcommand{\gz}{\zeta}
\newcommand{\vhat}[1]{\hat{\fat{#1}}}
\newcommand{\spc}{\vspace{0.7\baselineskip}}
\newcommand{\halfspc}{\vspace{0.3\baselineskip}}
\bibliographystyle{unsrt}
%\pagestyle{empty}
\newcommand{\twofig}[2]
 {
   \begin{figure}
     \begin{minipage}[t]{\minitwocolumn}
         \begin{center}   #1
         \end{center}
     \end{minipage}
         \hspace{\columnsep}
     \begin{minipage}[t]{\minitwocolumn}
         \begin{center} #2
         \end{center}
     \end{minipage}
   \end{figure}
 }
%%%%%%%%%%%%%%%%%%%%%%%%%%%%%%%%%
%\vspace*{\baselineskip}
\begin{flushright}
	Civil Engineering I\\
	May 14, 2020
\end{flushright}
\begin{center}
	{\LARGE \bf Lecuture Note 4 }
\end{center}
%\vspace{1.5cm}
\setcounter{section}{3}
\section{Design of Truss Structure}
\subsection{Axial Stress}
When desigining a truss structure, we need to know in advance the strength of 
the material. The strength of a truss member may be measured as the maximam axial 
force that the truss member can bear without breaking nor causing excessive deformation. 
The advantage of this definition is that we can evaluate the strength of the truss 
member just by performing a simple loading experiment.
On the other hand, a downside of the definition is that the strength depends on 
the thickness of the truss member. The larger the cross sectional area, the 
higher the strength is. This is very inconvenient for the strength should be a 
given parameter whereas the cross sectional area is one of the design 
variables. Dealing with the design criterion dependent on the design variables 
 makes the designing work rather tricky. 
 To avoid such a bizzar situation, we usually use the quantitiy called axial 
 stress as a measure of strength as well as the intensity of the axial force.
In the context of this lecture, the axial stress $\sigma$ is defined by
\begin{equation}
	\sigma=\frac{N}{A}
	\label{eqn:def_sigma}
\end{equation}
where $N$ is the axial force and $A$ is the cross sectional area of a truss member. 
The dimension of the axial stress is force per unit area. 
We take the positive axial stress as tensile and negative as compressional loading. 
The axial stress is a variable that depends on the applied force, 
support condition, geoemtry, and the thickness of the truss member. 
By performing a axial loading test on the truss member, 
we can identify the maximum tensile and compressional axial stress 
that the truss member can safely bears. 
We write the maximum compressional and tensile stress values as 
$\sigma_{a}^c$ and $\sigma_{a}^t$. 
The two maximum stress values are collectively called allowable stress.
The tensile and compressional allowable stress are usually differenet. 
For the sake of simplicity, we assume the two kinds of allowable stress agree 
each other and write
\begin{equation}
	\sigma_a^c=\sigma_a^t=\sigma_a.
	\label{eqn:sig_a}
\end{equation}
In designing a truss structure, we shall require 
\begin{equation}
	\left| \sigma \right| \leq \sigma_a
	\label{eqn:sig_a_cond}
\end{equation}
for every member of the truss. 
\subsection{Design of Simply Supported Truss -an example-}
Consider designing a simply supported Warren truss in Fig.\ref{fig:fig4_1} bridging 
a canal of width $3l$. To assemble the truss structure, we use the bars of equal 
length $l$ with a uniform circular cross section of diameter $d$ as illustrated in Fig.\ref{fig:fig4_2}-(a).
We suppose that the truss is used to support two straight water pipes (pipe 1 and 2) 
of identical cross sections. The truss provides an extra rigidity 
with the pipes and prevent the water-filled pipes from bending over the canal. 
We design the truss for this purpose based on the allowable stress concept.
In the structural analysis necessary for the design, we assume the followings.
\subsubsection{Assumptions}
\begin{enumerate}
\item
	The two pipes are identical and both are completely filled with water. 
\item
	The the truss members (bars) are so thin that their weight is negligible.
\item
	The weight of the water-filled pipes are supported evenly at the lower nodal 
		points (joints) A, A' to D,D' of the truss.
\item
	Due to the symmetry of the structure, the vertical force of magnitude 
	$F$ applied to the nodes A, B, C and D comes only from pipe 1.
\item
	The extent of the pipe that needed to be supported by the truss is $4l$. 
	You can imagine that the reminder of the weight goes to the supports on the 
	ground that are not shown in Fig.\ref{fig:fig4_1}.)
\item
	A structural member breaks (yields) if the axial stress (axial force/area) 
	exceeds a given allowable stress.
\end{enumerate}
\begin{figure}
	\begin{center}
	\includegraphics[width=1.25\linewidth]{fig4_1.eps} 
	\end{center}
	\caption{
		(a)front, (b) side, and (c) top views of a simply supported Warren truss consisting 
		of bars of uniform circular cross sections. 
		The truss bears the weight of two identical, symmetrically placed, completely filled water pipes.
		The weight is distributed evenly to the nodal points (joints) on the lower level 
		applying the downward vertical load of magnitude $F$ to the nodes A, B, C and D.
	}
	\label{fig:fig4_1}
\end{figure}
\begin{figure}
	\begin{center}
	\includegraphics[width=0.5\linewidth]{fig4_2.eps} 
	\end{center}
	\caption{
		The cross sections of (a) a truss member and (b) the pipe.  
	}
	\label{fig:fig4_2}
\end{figure}
\subsubsection{Structural analysis flow}
With the assumptions above, determine the minimum value of $d$ for which the 
truss can safely support the pipes without damaging (breaking) any of its members. 
The flow of structural analysis is as follows.
\begin{enumerate}
\item
	Determine the vertical reaction forces supplied to the truss by the supports at A and D. 
\item
	Determine the axial forces of truss member No.1 through 11. 
\item
	Determine the maximum tensile and compressional axial forces that arise in the truss members. 
\item
	%Let the outer diameter of the pipes be $D$, the thickness $t$, and the mass 
	%densities of the water and stainless steel be $\rho_w$ and $\rho_s$, 
	%respectively.
	Write the magnitude $F$ of the force in terms of the dimensions 
	$l,D,t$ of the bars and pipes. You may also use gravitiy constant $g$, 
	, the mass densities of water $\rho_w$ and steel $\rho_s$ 
	indicated in Fig.\ref{fig:fig4_2}. 
\item
	For given allowable stress $\sigma_a$, determine the minimum cross sectional 
	area $A_{min}$ and the the corresponding diameter $d_{min}$. 
%	Note that the axial stress is given by $N/A$ where $N$ is the axial force 
%	and $A=\pi d^2/4$ is the cross sectional area of the bar.
\item
	Evaluate $d_{min}$ numerically for\\
		\begin{center}
			$D$=500.0[mm], $t$=10.0[mm], 
			$\rho_s=$7.9[g/cm$^3$],
			$\rho_w=$1.0[g/cm$^3$],
			$l=3.0$[m],
			$\sigma_a$=20.0[kgf/mm$^2$], $g$=10.0[m/s$^2$]
		\end{center}
	If necessary, you may use a scientific calculator, spred sheet software, or computer programming languages. 
	%so that you can review the process of your calculation and make modification if necessary.
\end{enumerate}
\subsubsection{Worked out design calculations}
\begin{enumerate}
\item
\begin{equation}
	H_A=0, \ \ V_A+V_D=4F, 
	\label{eqn:}
\end{equation}
\begin{equation}
	V_D\times 3l - F\times l-F\times 2l-F\times 3l=0
	\label{eqn:}
\end{equation}
\begin{equation}
	V_A=V_B=2F, \ \ H_A=0
	\label{eqn:}
\end{equation}
\end{enumerate}
\end{document}
%%%%%%%%%%%%%%%%%%%%%%%%%%%%%%%%%%%%%%%%%%%%%%%%%%%%%%%%%%%%%%%%
