\documentclass[10pt,a4j]{article}
%\usepackage{graphicx,wrapfig}
\usepackage{graphicx}
\setlength{\topmargin}{-1.5cm}
%\setlength{\leftmargin}{1.5cm}
%\setlength{\textwidth}{15.5cm}
\setlength{\textheight}{25.2cm}
\newlength{\minitwocolumn}
\setlength{\minitwocolumn}{0.5\textwidth}
\addtolength{\minitwocolumn}{-\columnsep}
%\addtolength{\baselineskip}{-0.1\baselineskip}
%
\def\Mmaru#1{{\ooalign{\hfil#1\/\hfil\crcr
\raise.167ex\hbox{\mathhexbox 20D}}}}
%
\begin{document}
\newcommand{\fat}[1]{\mbox{\boldmath $#1$}}
\newcommand{\D}{\partial}
\newcommand{\w}{\omega}
\newcommand{\ga}{\alpha}
\newcommand{\gb}{\beta}
\newcommand{\gx}{\xi}
\newcommand{\gz}{\zeta}
\newcommand{\vhat}[1]{\hat{\fat{#1}}}
\newcommand{\spc}{\vspace{0.7\baselineskip}}
\newcommand{\halfspc}{\vspace{0.3\baselineskip}}
\bibliographystyle{unsrt}
%\pagestyle{empty}
\newcommand{\twofig}[2]
 {
   \begin{figure}
     \begin{minipage}[t]{\minitwocolumn}
         \begin{center}   #1
         \end{center}
     \end{minipage}
         \hspace{\columnsep}
     \begin{minipage}[t]{\minitwocolumn}
         \begin{center} #2
         \end{center}
     \end{minipage}
   \end{figure}
 }
%%%%%%%%%%%%%%%%%%%%%%%%%%%%%%%%%
%\vspace*{\baselineskip}
\begin{flushright}
	09/05/2019
\end{flushright}
\begin{center}
	{\Large \bf Civil Engineering I \\
	- Introduction to structural mechanics-\\
	Design of simply supported truss structure 2
	}  \\
\end{center}
%\vspace{1.5cm}
\section*{Exercises 4}
Consider designing a simply supported Warren truss shown in Fig.\ref{fig:fig4_1}.
To assemble the truss, we use the bars of length $l$ having a uniform circular 
cross section of diameter $d$ as illustrated in Fig.\ref{fig:fig4_2}-(a).
When completed, the truss is going to be used to support two straight pipes (pipe 1 and 2) 
crossing over a canal of width $3l$.
In performing the structural analysis necessary for the design, 
we will assume the followings.
\subsection*{Assumptions}
\begin{enumerate}
\item
	The two pipes are identical and are both completely filled with water. 
\item
	The the truss member (round bars) are so thin that their weight is negligible.
\item
	The weight of the water-filled pipes are supported evenly at the lower nodal 
	points (joints) of the truss.
\item
	Due to the symmetry of the structure, the vertical downward force of magnitude 
	$F$ applied to the nodes A, B, C and D comes only from pipe 1.
\item
	Although the pipes are infinitely long, the length of the pipe that needed to be 
	supported by the truss is $4l$ (Imagine that the reminder of the weight goes 
	to the supports on the ground that are not shown in Fig.\ref{fig:fig4_1}.)
\item
	A structural member breaks (yields) if the axial stress (axial force/area) 
	exceeds a certain value called an "allowable stress".
\end{enumerate}
\begin{figure}
	\begin{center}
	\includegraphics[width=1.25\linewidth]{fig4_1.eps} 
	\end{center}
	\caption{
		(a)A front, (b) side, and (c) top views of a simply supported Warren truss made 
		of round bars of uniform circular cross section. 
		The truss truss bears the weight of two identical, symmetrically placed, water filled pipes.
		The weight is distributed evenly to the nodal points (joints) on the lower level 
		applying the downward vertical load of magnitude $F$ to the nodes A, B, C and D.
	}
	\label{fig:fig4_1}
\end{figure}
\begin{figure}
	\begin{center}
	\includegraphics[width=0.5\linewidth]{fig4_2.eps} 
	\end{center}
	\caption{
		The cross sections of (a) a truss member (round bar) and (b) the pipe.  
	}
	\label{fig:fig4_2}
\end{figure}
\subsection*{Problems}
With the assumptions made above, determine the minimum value of $d$ for which the 
truss can safely support the pipes without damaging (breaking) any of its members. 
\begin{enumerate}
\item
	Determine the vertical reaction forces supplied to the truss by the supports at A and D. 
\item
	Determine the axial forces of truss member No.1 through 11. 
\item
	Determine the maximum tensile and compressional axial forces that arise in the truss members. 
\item
	%Let the outer diameter of the pipes be $D$, the thickness $t$, and the mass 
	%densities of the water and stainless steel be $\rho_w$ and $\rho_s$, 
	%respectively.
	Write the magnitude $F$ of the force using the dimensions 
	$l,D,t$ of the bars and pipes, gravitiy constant $g$, 
	and the mass densities of water $\rho_w$ and steel $\rho_s$ as 
	indicated in Fig.\ref{fig:fig4_2}. 
\item
	For given allowable stress $\sigma_a$, determine the minimum cross sectional 
	area $A_{min}$ and the the corresponding diameter $d_{min}$. 
	Note that the axial stress is given by $N/A$ where $N$ is the axial force 
	and $A=\pi d^2/4$ is the cross sectional area of the bar.
\item
	Evaluate $d_{min}$ numerically when\\
			$D$=500.0[mm], $t$=10.0[mm], 
			$\rho_s=$7.9[g/cm$^3$],
			$\rho_w=$1.0[g/cm$^3$],
			$l=3.0$[m],
			$\sigma_a$=20.0[kgf/mm$^2$], $g$=10.0[m/s$^2$]\\
	You may use a scientific calculator, but the use of Excel is recommended. 
	%so that you can review the process of your calculation and make modification if necessary.
\end{enumerate}
\end{document}
%%%%%%%%%%%%%%%%%%%%%%%%%%%%%%%%%%%%%%%%%%%%%%%%%%%%%%%%%%%%%%%%
