\documentclass[10pt,a4j]{article}
%\usepackage{graphicx,wrapfig}
\usepackage{graphicx}
\usepackage{showkeys}
\setlength{\topmargin}{-1.5cm}
%\setlength{\leftmargin}{1.5cm}
%\setlength{\textwidth}{15.5cm}
\setlength{\textheight}{25.2cm}
\newlength{\minitwocolumn}
\setlength{\minitwocolumn}{0.5\textwidth}
\addtolength{\minitwocolumn}{-\columnsep}
%\addtolength{\baselineskip}{-0.1\baselineskip}
%
\def\Mmaru#1{{\ooalign{\hfil#1\/\hfil\crcr
\raise.167ex\hbox{\mathhexbox 20D}}}}
%
\begin{document}
\newcommand{\fat}[1]{\mbox{\boldmath $#1$}}
\newcommand{\D}{\partial}
\newcommand{\w}{\omega}
\newcommand{\ga}{\alpha}
\newcommand{\gb}{\beta}
\newcommand{\gx}{\xi}
\newcommand{\gz}{\zeta}
\newcommand{\vhat}[1]{\hat{\fat{#1}}}
\newcommand{\spc}{\vspace{0.7\baselineskip}}
\newcommand{\halfspc}{\vspace{0.3\baselineskip}}
\bibliographystyle{unsrt}
%\pagestyle{empty}
\newcommand{\twofig}[2]
 {
   \begin{figure}
     \begin{minipage}[t]{\minitwocolumn}
         \begin{center}   #1
         \end{center}
     \end{minipage}
         \hspace{\columnsep}
     \begin{minipage}[t]{\minitwocolumn}
         \begin{center} #2
         \end{center}
     \end{minipage}
   \end{figure}
 }
%%%%%%%%%%%%%%%%%%%%%%%%%%%%%%%%%
%\vspace*{\baselineskip}
\begin{flushright}
	Civil Engineering I \\
	May 7, 2020
\end{flushright}
\begin{center}
	{\Large \bf Lecture Note 3 }
\end{center}
\setcounter{section}{2}
\section{Basics of Truss Structures}
\subsection{Introduction}
Bridges are fabricated by joining plates, bars, and cables of various sizes. The parts making up a bridge structure are called structural members. In this lecture, we focus on the mechanics of slender bar-like structural members. By "bar", we mean a slender object for which an evident longitudinal axis exists. The cross-sectional area of the bar may or may not be uniform in the axial direction. No assumption is made either regarding the shape of the cross-section. By joining bars, we can build a bridge structure called a truss. In this context, we often call the bars as truss members. In Fig.\ref{fig:fig1}, examples of truss structures are shown. The truss structures in this figure bear the name of their inventors. Truss structures are characterized by a triangular assemblage of the members. The major advantage of the truss structure is weights. Trusses are often lighter in weight than girder bridges, which are the most prevalent bridge structure best suited to short-span bridges. In designing truss structures, we need to address the following issues.
%DONE
\begin{itemize}
\item
	How is the whole truss structure supported?
\item
	How to connect two or more truss members at the joints?
\item
	How much force each truss member has to bear?
\end{itemize}
Those are the topics to be discussed here in some detail. 
\begin{figure}[h]
	\begin{center}
	\includegraphics[width=1.0\linewidth]{fig1.eps} 
	\end{center}
	\caption{Examples of truss structures. (a) Warren truss, (b) Howe truss, and (c) Pratt truss.} 
	\label{fig:fig1}
\end{figure}
\subsection{Support Condition and Reaction Forces}
Three support conditions that are basic in structural mechanics are 
\begin{itemize}
\item
	fixed (or clamped) support,
\item
	pin (or hinge) support, and 
\item
	roller support.
\end{itemize}
For each of the three support conditions, we will look at the constraints that the support imposes to a body. In doing so, we need not be specific about the geometry of the supported object. For simplicity, therefore, we'll study the interaction between the supports and a single straight bar. 
\subsubsection{Fixed support}
When a bar is fixed to a rigid and immobile wall as shown in Fig.\ref{fig:fig3}-(a),
we say the bar is held by a "fixed" or "clamped" support. By the fixed support, we assume that the translational and rotational motion is completely restrained. In the setup shown in Fig.\ref{fig:fig3}-(a), the horizontal, vertical, and rotational displacements at the left end of the bar are therefore not allowed at all. To see how the fixed support works, suppose that the bar is under a vertical force as illustrated in Figure 3-a. In reaction to the loading, the support applies a force and probably a moment as well. The force and moment applied by a support are called reaction force and reaction moment, respectively. The reaction force and moment are required to statically support the bar by equilibrating the total forces and moments. To see this, assume that the bar is subjected to the reaction forces and moment as shown in Fig.\ref{fig:fig3}-(b)). Then the equilibrium conditions for the horizontal and vertical forces are written as 
\begin{equation}
	H=0, \ \ F-V=0, 
	\label{eqn:}
\end{equation}
and for the moment about A as 
\begin{equation}
	M+F\times l =0.
	\label{eqn:}
\end{equation}
As a result, the equilibrium conditions can be satisfied if and only if the reaction forces such as   
\begin{equation}
	H=0, \ \ V=F, \ \ M=-Fl. 
	\label{eqn:}
\end{equation}
are applied to the bar. are applied to the bar. In the foregoing discussion, however, we have not shown how to generate the reaction moment $M$ independently from the reaction force. The simplest solution is to apply a couple of forces as shown in Fig.\ref{fig:fig6} where $K$ and $\varepsilon$ are constants such that 
\begin{equation}
	M=K\varepsilon.
	\label{eqn:}
\end{equation}
Note that the sum of the two forces consisting of the couple vanishes. 
This means we can generate a pure moment of arbitrary magnitude. 
For the given $M$, however, we cannot determine $K$ and $\varepsilon$ uniquely. 
Thus we have to be content with specifying only $M$ rather than $F$ and $\varepsilon$. 
This is why we have indicated the reaction moment in Fig.\ref{fig:fig3} instead of a couple of forces.
\begin{figure}[h]
	\begin{center}
	\includegraphics[width=0.8\linewidth]{fig3.eps} 
	\end{center}
	\caption{(a) A rigid and immobile wall as a fixed support for a bar. 
	(b) The reaction forces and moment from the fixed support.} 
	\label{fig:fig3}
\end{figure}
\begin{figure}[h]
	\begin{center}
	\includegraphics[width=0.4\linewidth]{fig6.eps} 
	\end{center}
	\caption{Generation of a pure moment by a couple of forces.} 
	\label{fig:fig6}
\end{figure}
\subsubsection{Pin support}
The pin support is a fixture with a hinge that pins the bar as shown in Fig.\ref{fig:fig7}. 
Around the hinge, the bar may rotate freely and smoothly. 
Since the support itself is fixed to an immobile floor or a wall, the bar is not allowed to move at the hinge in any directions.  
This means the pin supports apply the horizontal and vertical reaction forces in reaction to external loading. 
On the other hand, the reaction moment does not arise since the hinge does not prevent the bar from rotating. 
\begin{figure}[h]
	\begin{center}
	\includegraphics[width=0.6\linewidth]{fig7.eps} 
	\end{center}
	\caption{Pin supports installed on a floor and wall.} 
	\label{fig:fig7}
\end{figure}
\subsubsection{Roller support}
Figure \ref{fig:fig8} illustrates roller supports installed on a floor and a wall. 
As depicted in this figure, the roller supports consist of a hinge and a roller. 
The hinge allows the bar to rotate freely as the pin supports do. 
The roller allows the pin-supported bar to slide freely along the wall or the floor. 
When the roller support is installed on a vertical wall, the vertical displacement may take place at the support whereas  the horizontal displacement may not. In this case, therefore, the roller support generates only the horizontal reaction force. 
When the roller support is installed on a horizontal floor, the support prevents the vertical displacement whereas the horizontal displacement may take place freely. This means the roller support works as a source of vertical reaction force.
\begin{figure}[h]
	\begin{center}
	\includegraphics[width=0.6\linewidth]{fig8.eps} 
	\end{center}
	\caption{The roller supports installed on an immobile floor and a wall.} 
	\label{fig:fig8}
\end{figure}
\subsubsection{Simple support}
The combination of the roller and the pin support is called simple support. 
When a bar is supported by the pin support at one end and by the roller support at the other end, we say the bar is simply supported. When a bar is simply supported, we can determine the reaction forces based on the equilibrium conditions for the force and moment. To see this, consider a simply supported bar shown in Fig.\ref{fig:fig3_2} subjected to an angled load of magnitude $F$. The pin support on the left applies both vertical and horizontal reaction forces since the displacements in those directions are constrained. At the right end of the bar, rotation and lateral displacement are allowed. Thus, the roller support on the right applies only a vertical reaction force. Fig.\ref{fig:fig3_1} shows the non-vanishing reaction forces and the external force at the center of the bar. The diagram showing the forces acting to a body or a portion of the body is called a free body diagram. Fig.\ref{fig:fig3_1} is an example of the free body diagram for a simply supported bar.
\begin{figure}[h]
	\begin{center}
	\includegraphics[width=0.6\linewidth]{fig3_2.eps} 
	\end{center}
	\caption{A simply supported bar AC subjected to an angled force of magnitude $F$.}
	\label{fig:fig3_2}
\end{figure}
\begin{figure}[h]
	\begin{center}
	\includegraphics[width=0.6\linewidth]{fig3_1.eps} 
	\end{center}
	\caption{
		A free body diagram for a simply supported bar.
		$F, H_A, V_A, V_C$ and $|\fat{f}|=F$ denote the magnitude of forces.
	}
	\label{fig:fig3_1}
\end{figure}
Referring to the free body diagram, we can write the equilibrium equations.
Regarding the moment about $A$, we have  
\[
	-F\sin\frac{\pi}{4}\times \frac{l}{2} + V_C\times l =0.
\]
Thus $V_C=\frac{F}{2\sqrt{2}}$ follows. 
The equilibrium of the horizontal force gives
\[
	H=F\cos\frac{\pi}{4}=\frac{F}{\sqrt{2}}, 
\]
while that of the vertical force requires 
\[
	F\sin\frac{\pi}{4} -(V_A+V_C)=0
\]
to give $V_A$ as 
\[
	V_A=\frac{F}{2\sqrt{2}}.
\]
\subsection{Pin Connection}
A point where two or more truss members are joined together is called a joint or a node. In truss structures, the members are connected at a joint either by a gusset plate or a hinge as depicted in Fig.\ref{fig:fig4}. The former is called a rigid connection while the latter a pin connection. At a rigidly connected joint, the truss members are joined tightly by a gusset plate that the angles between the truss members do not change upon loading. Hence the force and moment are both transmitted across the rigid joints from one member to the other. At a pin connection, on the other hand, the truss members can rotate freely about the joint without generating a moment. This means the pin-connected joints do not bear a moment, and only a force can be transmitted across the joints. 
\begin{figure}[h]
	\begin{center}
	\includegraphics[width=0.8\linewidth]{fig4.eps} 
	\end{center}
	\caption{A rigid and pin connections joining two truss members.} 
	\label{fig:fig4}
\end{figure}

When a truss structure is externally loaded, the force and moment are transmitted through the structure. As a result, every truss member bears a force and moment at the joints even if the external load is not applied directly to it. The load-bearing mechanism of the truss structure changes significantly whether the members are joined rigidly or pin-connected. To see this, let's first examine a truss member with rigidly connected ends. For this purpose, we denote the moment, the axial, and transverse force acting at the bar's ends as shown in Fig.\ref{fig:fig5}-(a). The six components of the nodal forces and moments are mutually not independent since the equilibrium conditions must be satisfied.
Regarding the axial and transverse forces, we have 
\begin{equation}
	N_A-N_B=0, \ Q_A-Q_B=0.
	\label{eqn:equib_NQ}
\end{equation}
Equation \ref{eqn:equib_NQ} enable us to introduce independent axial and transverse force components 
$N$ and $Q$ as
\begin{equation}
	N=N_A=N_B,\  Q=Q_A=Q_B.
	\label{eqn:def_NQ}
\end{equation}
Moreover, the equilibrium condition for the moment about $A$ gives 
\begin{equation}
	M_A-M_B+Q_Bl=0 
	\label{eqn:equib_Mab}
\end{equation}
reducing the number of independent components to three (i.e. $N,Q$ and $M_A$ or $M_B$).\\

When the truss member $AB$ is pin-connected at both ends, the 
number of independent components of the nodal force and moment can be reduced further. 
By assumption, the pin connected joints do not bear moment. Hence   
\begin{equation}
		M_A=M_B=0, 
	\label{eqn:}
\end{equation}
which reduces eq.(\ref{eqn:equib_Mab}) to $Q=0$.
In conclusion, we have 
\begin{equation}
	N_A=N_B=N,\  Q_A=Q_B=0, \  M_A=M_B=0.
	\label{eqn:sec_forces}
\end{equation}
for pin connected truss members.  
%This means that the truss members transmit only the axial forces if they are connected at both ends by hinges. 
As we shall see in the next section, eq.(\ref{eqn:sec_forces}) greatly simplifies the analyses of truss structures. 
\begin{figure}[h]
	\begin{center}
	\includegraphics[width=0.8\linewidth]{fig5.eps} 
	\end{center}
	\caption{The nodal force $N$ applied to a truss member.$N$ is called axial force.}
	\label{fig:fig5}
\end{figure}
%
%
\subsection{Analysis of Truss Structure}
One of the primary objectives of structural analysis of a truss is to determine the axial forces when the truss is externally loaded. Evaluating the axial forces is important because an excessive axial force leads to a failure of the member and/or the structure. When a pin-connected truss is simply supported, the axial forces can be determined by imposing the equilibrium conditions node by node. There are other approaches to this problem. However, the node-by-node method is the simplest although not the most efficient in many cases. To learn the node-by-node approach, let's consider obtaining the axial forces in the truss shown in Fig.\ref{fig:fig9}.\\ 

\begin{figure}[h]
	\begin{center}
	\includegraphics[width=0.4\linewidth]{fig9.eps} 
	\end{center}
	\caption{A simply supported truss.} 
	\label{fig:fig9}
\end{figure}
Structural analysis of the truss starts with the identification of the external forces including reaction forces. 
In the present example, the horizontal force at node $C$ of magnitude $F$ is the given external load. 
The reaction forces are not readily available. However, we have already learned that the equilibrium conditions determine the reaction forces for a simply supported bar. 
Similarly, the equilibrium conditions for the whole truss structure determine the reaction forces for simply supported trusses. 
To take this approach, we first draw a free body diagram as shown in Fig.\ref{fig:fig9}-(a). 
Note that $H_A$, $V_A$, and $V_B$ are the magnitude of reaction forces. 
The arrows indicate the direction when the force vector components take positive values. 
The equilibrium equations of the forces and moment are then written as follows.
\begin{equation}
	F+H_A=0, \ \ V_A+V_B=0, \ \ -F\times l\sin\frac{\pi}{3}+V_B\times l=0 
	\label{eqn:equib_whole}
\end{equation}
Thus we have 
\begin{equation}
	H_A=-F, \ \ V_A=-V_B=\frac{\sqrt{3}}{2}F
	\label{eqn:reac_frc}
\end{equation}
Note that the total moment about $A$ has been considered in the third equation of (\ref{eqn:equib_whole}).

Given the reaction forces, we are ready to determine the axial forces. 
To this end, it is convenient to number the truss members as shown in Fig.\ref{fig:fig8}-(a), 
and denote the axial forces of truss member $i$ as $N_i$ where $i$ assumes 1,2, or 3. 
In the analysis to follow, it is important to note that the tensile axial force is taken to be positive while the compressional force to be negative. 
If a truss member is bearing a tensile axial force, therefore, we can tell that the member is being pulled by the joints on its ends. By the law of action and reaction (Newton's 3rd law), this implies that the truss member under the tensile force is pulling the joints. To make the discussion clearer, let's look at the truss member 1. As shown in \ref{fig:fig10}-(a), truss member 1 is connected to node $A$ and $B$. 
Hence, it pulls the nodes $A$ and $B$ together if $N_1>0$ or pushes them apart if $N_1<0$. 
The same observation applies to the truss members 2 and 3. 
Thus we can illustrate the forces acting to each node as shown in Fig.\ref{fig:fig9}-(b). 
It should be noted here that Fig.\ref{fig:fig10} depicts the forces acting to the node, which is why the truss members are indicated by dashed lines there. By imposing the equilibrium conditions for horizontal and vertical forces node by node, we can determine the axial forces. This procedure goes as follows.
\begin{enumerate}
\item
	Node $A$: \\
	The equilibrium equations for the horizontal and vertical forces at 
	node $A$ are written respectively as 
	\begin{equation}
		N_1+H_A+N_2\cos\alpha=0, \ \ N_2\sin\alpha+V_A=0
		\label{eqn:nodeA}
	\end{equation}
	where $\alpha=\frac{\pi}{3}$. Solving eqs.(\ref{eqn:nodeA}), 
	we have $N_1=\frac{F}{2}$ and $N_2=F$. 
\item
	Node $B$: \\
	The equilibrium condtions for node $B$ written as 
	\begin{equation}
		N_1+N_3\cos\alpha=0, \ \ N_3\sin\alpha+V_B=0
		\label{eqn:nodeB}
	\end{equation}
	give $N_1=\frac{F}{2}$ and $N_3=-F$. We can verify that the axial 
	force of member 1 due to eqs.(\ref{eqn:nodeA}) agrees with that of eq.(\ref{eqn:nodeB}).
\item
	Node $C$: \\
	The equilibrium conditions for node $C$ are written as 
	\begin{equation}
		F-N_1\cos\alpha+N_3\cos\alpha=0, \ \ N_2\sin\alpha+N_3\sin\alpha=0.
		\label{eqn:nodeC}
	\end{equation}
	Since we have already determined all the axial forces, eqs.(\ref{eqn:nodeC}) are redundant.  
	However, it is a good practice to verify the solutions using redundant 
	but independently derived relations that should be satisfied.  
	In this problem, we can verify that our solutions $N_1,N_2$ and $N_3$ do 
	satisfy eq.(\ref{eqn:nodeC}).
\end{enumerate}
As it has been shown, two equations are available for every nodal point, which means we can determine up to two axial forces per node. If more than two members are connected to a node, it is not possible to determine the relevant axial forces all at once. In that case, we should postpone imposing equilibrium conditions to that node until at least one of the axial forces involved is determined as a result of the analysis on neighboring nodes.
\begin{figure}[h]
	\begin{center}
	\includegraphics[width=0.8\linewidth]{fig10.eps} 
	\end{center}
	\caption{(a) A free body diagram for the truss of Fig.\ref{fig:fig9}. 
	(b) The forces acting to the nodal points of the truss}
	\label{fig:fig10}
\end{figure}
\subsection{Solved Problems}
\begin{enumerate}
\item
Obtain the reaction forces for a simply supported bar shown in Fig.\ref{fig:fig3_3}. 
\begin{figure}[h]
	\begin{center}
	\includegraphics[width=0.5\linewidth]{fig3_3.eps} 
	\end{center}
	\caption{A simply supported bar subjected to a vertical force of magnitude $F$.}
	\label{fig:fig3_3}
\end{figure}
{\small
	A free body diagram for the simply supported bar is shown in Fig.\ref{fig:fig15}. 
	The horizontal and vertical components of the reaction forces are denoted in
	this figure as $H_A, V_A$, and $V_B$.
	The equilibrium equations for the reaction forces are written as 
	\[
		H_A=0, \ \ V_A+V_B=F, 
	\]
	and 
	\[
		-F\times a +V_B\times l =0
	\]
	where the last equation is concerned with the moment about $A$.
	From the equations above, we obtain
	\[
		H_A=0, \ \ V_A=\frac{b}{l}F, \ \ V_B=\frac{a}{l}F.
	\]
}
	\begin{figure}[h]
	\begin{center}
	\includegraphics[width=0.5\linewidth]{fig15.eps} 
	\end{center}
	\caption{A free body diagram for the bar shown in Fig.\ref{fig:fig3_3}.} 
	\label{fig:fig15}
	\end{figure}
\item
Examine the two support conditions shown in Fig.\ref{fig:fig3_4}, 
and determine whether all the reaction forces and moments can 
be determined by the equilibrium equations. 
\begin{figure}[h]
	\begin{center}
	\includegraphics[width=0.8\linewidth]{fig3_4.eps} 
	\end{center}
	\caption{A straight bar in two support conditions.} 
	\label{fig:fig3_4}
\end{figure}
{\small
	The free body diagrams for the two support conditions are shown in Fig.\ref{fig:fig16}. 
	As shown in this figure, the number of unknown reaction force components is  
	6 for (a) and 4 for (b) whereas the number of equilibrium equations 
	available is three. Hence, the equilibrium conditions give an underdetermined 
	system of equations, and it is not possible to determine the 
	reaction forces uniquely. When reaction forces can be determined by the equilibrium 
	equations, we say the structure is statically determinate. Otherwise, the structure 
	is called statically indeterminate. The structures shown in Fig.\ref{fig:fig3_4} are 
	 examples of statically indeterminate structures.
}
\begin{figure}[h]
	\begin{center}
	\includegraphics[width=0.8\linewidth]{fig16.eps} 
	\end{center}
	\caption{Free body diagrams for the bars shown in Fig.\ref{fig:fig3_4}.} 
	\label{fig:fig16}
\end{figure}
\item
	Obtain the reaction and axial forces acting to the simply supported truss 
	shown in Fig.\ref{fig:13}.
\begin{figure}[h]
	\begin{center}
	\includegraphics[width=0.5\linewidth]{fig13.eps} 
	\end{center}
	\caption{A simply supported truss. } 
	\label{fig:fig13}
\end{figure}
{\small 
	We assign a unique number to each truss member, and draw a free body diagram 
	as shown in Fig.\ref{fig:fig14}-(a) to obtain the reaction forces in the first place. 
	If we notice that the line of action of the force passes node $A$, the equilibrium 
	of the moment about $A$ is given immediately as  
	\[
		V_B\times l =0. \ \ \Rightarrow V_B=0
	\]
	Then, the equilibrium equations: 
	\[ 
		H_A+F\cos\alpha =0 , \ \ V_A+V_B+F\sin\alpha=0, \ \ \left( \alpha=\frac{\pi}{4}\right)
	\]
	for the horizontal and vertical reaction forces give
	\[
		H_A=V_A=-\frac{F}{\sqrt{2}}, \ \ V_B=0.
	\]

	To determine the axial forces one by one, we draw a diagram 
	that shows the forces acting to each nodes(see Fig.\ref{fig:fig14}). 
	In imposing the equilibrium conditions going through the nodes one by one, 
	we need to start from the node to which two truss members are connected.
	If we start from node $D$, the equilibrium conditions give
	\[ 
		N_4=N_5=\frac{F}{\sqrt{2}}.
	\]
	If we start from node $A$ instead, we obtain
	\[
		N_1=-H_A=\frac{F}{\sqrt{2}}, \ \ 
		N_2=-V_A=\frac{F}{\sqrt{2}}
	\]
	Either way, the equilibrium equations for node $C$ and $D$ are made 
	solvable as a result.
	For the node $C$, the equilibrium equations are given by 
	\[
		N_5+\frac{N_3}{\sqrt{2}}=0, \ \ 
		N_2+\frac{N_3}{\sqrt{2}}=0
	\]
	whereas for node $D$ by
	\[
		N_4+\frac{N_3}{\sqrt{2}}+V_B=0, \ \ 
		N_1+\frac{N_3}{\sqrt{2}}=0.
	\]
	From the set of equations presented above, we obtain $N_3=-F$.
}
\begin{figure}[h]
	\begin{center}
	\includegraphics[width=1.0\linewidth]{fig14.eps} 
	\end{center}
	\caption{(a)A free body diagram for the whole truss structure.
	(b) Axial forces acting to the pin-connected joints.} 
	\label{fig:fig14}
\end{figure}
\end{enumerate}
\subsection{Supplementary Problems}
\begin{enumerate}
\item
For a bar subjected to two angled forces shown in Fig.\ref{fig:fig11}, 
determine the reaction forces and moment at the fixed support.
\begin{figure}[h]
	\begin{center}
	\includegraphics[width=0.5\linewidth]{fig11.eps} 
	\end{center}
	\caption{A bar subjected to two angled forces of equal magnitude $F$.}
	\label{fig:fig11}
\end{figure}
\item
	Examine the bars shown in Fig.\ref{fig:fig12} under a vertical load 
	of magnitude $F$, and answer whether the reaction forces and 
	moment can be determined by the equilibrium conditions.
	If the answer is positive, obtain the reaction forces and moment. 
\begin{figure}[h]
	\begin{center}
	\includegraphics[width=1.0\linewidth]{fig12.eps} 
	\end{center}
	\caption{Bars subjected to a vertical force of magnitude $F$.} 
	\label{fig:fig12}
\end{figure}
\item
Obtain the reaction and axial forces acting to the truss structures shown in Fig.\ref{fig:fig3_5}.
\begin{figure}[h]
	\begin{center}
	\includegraphics[width=1.0\linewidth]{fig3_5.eps} 
	\end{center}
	\caption{Simply supported truss structures.} 
	\label{fig:fig3_5}
\end{figure}
\end{enumerate}
\end{document}
%%%%%%%%%%%%%%%%%%%%%%%%%%%%%%%%%%%%%%%%%%%%%%%%%%%%%%%%%%%%%%%%
