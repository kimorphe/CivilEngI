\documentclass[10pt,a4j]{jarticle}
%\usepackage{graphicx,wrapfig}
\usepackage{graphicx}
\setlength{\topmargin}{-1.5cm}
\setlength{\textwidth}{16.5cm}
\setlength{\textheight}{25.2cm}
\newlength{\minitwocolumn}
\setlength{\minitwocolumn}{0.5\textwidth}
\addtolength{\minitwocolumn}{-\columnsep}
%\addtolength{\baselineskip}{-0.1\baselineskip}
%
\def\Mmaru#1{{\ooalign{\hfil#1\/\hfil\crcr
\raise.167ex\hbox{\mathhexbox 20D}}}}
%
\begin{document}
\newcommand{\fat}[1]{\mbox{\boldmath $#1$}}
\newcommand{\D}{\partial}
\newcommand{\w}{\omega}
\newcommand{\ga}{\alpha}
\newcommand{\gb}{\beta}
\newcommand{\gx}{\xi}
\newcommand{\gz}{\zeta}
\newcommand{\vhat}[1]{\hat{\fat{#1}}}
\newcommand{\spc}{\vspace{0.7\baselineskip}}
\newcommand{\halfspc}{\vspace{0.3\baselineskip}}
\bibliographystyle{unsrt}
%\pagestyle{empty}
\newcommand{\twofig}[2]
 {
   \begin{figure}
     \begin{minipage}[t]{\minitwocolumn}
         \begin{center}   #1
         \end{center}
     \end{minipage}
         \hspace{\columnsep}
     \begin{minipage}[t]{\minitwocolumn}
         \begin{center} #2
         \end{center}
     \end{minipage}
   \end{figure}
 }
%%%%%%%%%%%%%%%%%%%%%%%%%%%%%%%%%
%\vspace*{\baselineskip}
\begin{center}
	{\Large \bf Lecture Note 1, Civil Engineering I - an introduction to structural mechanics}  \\
\end{center}
%%%%%%%%%%%%%%%%%%%%%%%%%%%%%%%%%%%%%%%%%%%%%%%%%%%%%%%%%%%%%%%%
\section{Review of Vector Algebra}
\subsection{Scalar and vector quantites}
Scalar is a quantitiy that has only a magnitude.
The magnitude  can either be positive or negative. 
The examples of scalar quantities appearing in physics are 
mass, temperature, concentration, length, weight and so on.
Vector is a quantity that has both a magnitude and direction. 
The examples are displacement, velocity, acceleration, 
force, heat flux, mass flux, electrip poloarization, etc. 
Graphically, a vector is represented by an arrow. 
When drawing the arrow, the magnitude and directions of the vector 
are indicated by the length and the orientation of the arrow head, 
respectively. For mathematical manipulations of vectors, symbolic 
representations are far more convenient. 
To denote vector quantities in typed manuscripts symbolically, 
boldface letters such as $\fat{x},\fat{u},\fat{v},\dots$ 
are often used instead of writting simply as $x, u$ and $v$.
This is to distinguish vectors from scalar qunantities at sight. 
To denote the magnitude of a vector $\fat{a}$, we either 
place vertical bars on both sides of $\fat{a}$ and write 
as $|\fat{a}|$,or simply as $a$.
If $\fat{a}$ has a unit magnitude, namely $|\fat{a}|=a=1$, 
$\fat{a}$ is said to be a unit vector. 
\subsection{Vector Identity}
Two vectors, say $\fat{a}$ and $\fat{b}$, are considered identical 
if their magnitude and direction are the same. 
In that case, we write $\fat{a}=\fat{b}$ regardless of their locations. 
If a pair of forces, say $\fat{f}$ and $\fat{g}$ of equal magnitude and 
direction $(|\fat{f}|=|\fat{g}|)$ are acting on two different points 
on a body, $\fat{f}$ and $\fat{g}$ are physically different but 
mathematically identical. 
\subsection{Basic Algebraic Operation on Vectors}
\subsubsection{Scalar multiplication}
A multiplication between a scalar $s$ and a vector $a$ are called 
"scalar multiplication" and is written as $s\fat{a}$. 
Scalar multiplication is an operation that gives us a new vector $s\fat{a}$ 
whose magnitude is as $|s|$ times greater ($|s|>1$) or smaller $(|s|<1)$ than 
$|\fat{a}|$.
When $s>0$, $s\fat{a}$ points in the same direction as $\fat{a}$. 
When $s<0$, $s\fat{a}$ points in the opposite direction to $\fat{a}$. 
If $s=1$, then the multiplication does vertiually nothing to $\fat{a}$, thus 
$1\fat{a}=\fat{a}$. When $s=0$, $s\fat{a}$ is equal to zero vector denoted by $\fat{0}$. 
Zero vector is a special vector whose length is zero. 
The direction of $\fat{0}$ is indeterminate. 
\subsubsection{Vector addition}
Vector addition is an operation that produces a new vector by connecting 
a pair of vectors. 
When we have two vectors, say $\fat{a}$ and $\fat{b}$, 
we can form a directed broken line by connecting the tail of one vector with 
the head of the other. The sum $\fat{a}+\fat{b}$ is defined to be a vector 
extending from the tail to the head of the broken line.
Note that vector addition is "commutative" since 
\[
	\fat{a}+\fat{b}=\fat{b}+\fat{a}
\]
alwyas hold true. This means the order of summation is immaterial. 
Clearly, we can add more than two variables by summing two vectors, repeatedly. 
For example, we can perform summation of three vectors either by 
\[
	\fat{a}+\left( \fat{b}+\fat{c} \right)
\]
or by 
\[
	\left( \fat{a}+\fat{b} \right) + \fat{c}
\]
We can prove graphically that the addition by result in a same vector. 
\subsection{Inner product} 
\subsection{Exterior product} 
When a pair of vectors $\fat{a}$ and $\fat{b}$ make a right 
angle, they are said to be mutually orothogonal.  
\end{document}
\begin{equation}
	\overline{\varepsilon}(a,b) =\frac{\Delta l (a,b)}{b-a}
	\label{eqn:eps_bar}
\end{equation}
このとき,$b \rightarrow a$の極限:
\begin{equation}
	\varepsilon(a)=\lim_{b \rightarrow a}\overline{\varepsilon}(a,b)
	\label{eqn:eps_def}
\end{equation}
を, 位置$x=a$のひずみと呼び, $x=a$における変形を表す量として用いる. 
$\varepsilon(a)$は, $x$における変位を$u(x)$と表すとき,
\begin{equation}
	\Delta l(a,b)=u(b)-u(a)
	\label{eqn:dell_u}
\end{equation}
であることから, $u$と$\varepsilon$の間には
\begin{equation}
	\varepsilon(a) = \lim _{b\rightarrow a} \frac{u(b)-u(a)}{b-a}= \left.\frac{du}{dx}\right|_{x=a}
	\label{eqn:eps_u}
\end{equation}
の関係がある.従って, ひずみと変位の関係は
\begin{equation}
	\varepsilon=\frac{du}{dx}
	\label{eqn:e_dudx}
\end{equation}
あるいは
\begin{equation}
	u(b)-u(a)=\int_{x'=a}^{x'=b} \varepsilon(x') dx'
	\label{eqn:u_int_eps}
\end{equation}
と表される.もし, $u(a)=0$となるように$a$を選ぶことができるならば, 
$b=x$と置き換え, $a$をそのように選ぶことで 式(\ref{eqn:u_int_eps})を
\begin{equation}
	u(x)=\int_{x'=a}^{x'=x} \varepsilon(x') dx'
	\label{eqn:u_int_eps0}
\end{equation}
とすることができる.
\paragraph{問題}
\begin{enumerate}
\item
長さ$l$の棒部材に,図\ref{fig:fig1}に示すような変位が発生しているとする.
このとき, ひずみ$\varepsilon(x)$を求め, その結果をグラフとして示せ.
ただし, 図中の$x$は棒部材の軸方向に沿って取った座標を表し, その原点$x=0$
は棒部材の一方の端部に位置するものとする.
\item
図\ref{fig:fig2}のように, 一端が固定された長さ$l$の棒部材に,
同図(a)〜(c)に示すひずみが発生しているとする.このとき, 変位$u(x)$を求め, 
その結果をグラフとして示すとともに, 区間$(0,l)$, $(0,l/2)$および$(l/2,l)$に
おける伸びを, (a)から(c)それぞれのひずみ分布に対して求めよ.
\end{enumerate}
\begin{figure}[h]
	\begin{center}
	\includegraphics[width=0.9\linewidth]{ex1_disp.eps} 
	\end{center}
	\caption{棒部材中の変位分布を表すグラフ.} 
	\label{fig:fig1}
\end{figure}
\begin{figure}[h]
	\begin{center}
	\includegraphics[width=0.9\linewidth]{ex2_strain.eps} 
	\end{center}
	\caption{棒部材中のひずみ分布を表すグラフ.} 
	\label{fig:fig2}
\end{figure}
%%%%%%%%%%%%%%%%%%%%%%%%%%%%%%%%%%%%%%%%%%%%%%%%%%%%%%%%
\newpage
\subsection{力に関する諸量}
棒部材の長手方向に働く, 単位長さあたりの外力を$p(x)$と表し, これを分布力
(正確には分布外力)あるいは物体力と呼ぶ. 区間$(a,b)$に作用する分布力の合力を
$P(a,b)$と表すとき, $P(a,b)$は
\begin{equation}
	P(a,b)=\int_{x'=a}^{x'=b}p(x')dx'=P(b)-P(a)
	\label{eqn:Pab}
\end{equation}
で得られる.ただし, 式(\ref{eqn:Pab})最右辺の$P(x)$は, $p(x)$の
任意の原始関数を表す.

位置$x$における軸力を$N(x)$, 断面積を$A(x)$とする.
軸力は外力に対する応答として部材内に発生する内力の一種である.
外力と内力は, 部材が静止状態にあるとき, 任意の区間で釣り合っている
必要がある. よって, 
任意の$0<a<b<l$に対して
\begin{equation}
	P(a,b)+N(b)-N(a)=0
	\label{eqn:Pab_equib}
\end{equation}
が成り立つ.式(\ref{eqn:Pab_equib})において$a=x, b=x+\Delta x$として, 
両辺を$\Delta x$で割り, その後$\Delta x\rightarrow 0$の極限をとれば
\begin{equation}
	\frac{P(x+\Delta x)-P(x)}{\Delta x}+\frac{N(x+\Delta x)-N(x)}{\Delta x}=0 \,\,
	\rightarrow
	\, \,
	\frac{dP}{dx}+\frac{dN}{dx}=0
	\label{eqn:Pab_equib_lim}
\end{equation}
となる. さらに,$P'(x)=p(x)$を用いれば式(\ref{eqn:Pab_equib_lim})より
\begin{equation}
	\frac{dN}{dx}+p=0
	\label{eqn:Nx_equib}
\end{equation}
が得られ, これも軸力の釣り合い式を表す. 
1次元軸力問題では部材断面に垂直に働く単位面積あたりの力:
\begin{equation}
	\sigma(x) = \frac{N(x)}{A(x)}
	\label{eqn:sigma1D_def}
\end{equation}
を応力と呼ぶ. 従って, 応力の釣り合い方程式は式(\ref{eqn:sigma1D_def})と
式(\ref{eqn:Nx_equib})より
\begin{equation}
	\frac{d\sigma A}{dx}+p=0
	\label{eqn:sigx_equib}
\end{equation}
となる.
\paragraph{力の単位}
科学,工学の分野では通常MKS単位系で各種の量が表される.すなわち, 
長さはメートル[m], 質量はキログラム[kg], 時間は秒[s]が単位として用いられる.
力は,ニュートンの第二法則によれば,質量×加速度の次元を持つことから, 
その単位は[kg$\cdot$m/s$^2$]であり,これをニュートン[N]と呼ぶ.
これに対して日常生活では,キログラム重[kgf]が力の単位としてしばしば
用いられる.キログラム重は, 質量1[kg]の物体が地球上で受ける重力の
大きさとして定義される.従って,Nとkgの換算は
\begin{center}
	1[kgf]=1[kg] $\times$ (重力加速度  $g\simeq$=9.8[m/s$^2$])=9.8[N]
\end{center}
で行うことができる.一方,応力や圧力の次元は,単位面積あたりの力である.よって
その単位は[N/m$^2$]であり,これをパスカル[Pa]と呼ぶ.
問題となる数量が非常に大きい,あるいは小さいとき,
その数値を適当な接頭辞を用いて表すことが多い.例えば,
10$^6$[Pa]は1[MPa](メガパスカル), 10$^{-6}$[s]は1[$\mu$s](マイクロ秒)等と
書くことができる.
\paragraph{問題}
図\ref{fig:fig3}に示すような, 左端が固定された, 長さ$l$の棒部材に, 
外力が作用している.このとき, 部材内部に発生する軸力$N$の分布を求め, 
その結果をグラフとして示せ.また, 棒部材が固定壁から受ける力(反力)を
(a)から(c)それぞれの場合について求めよ.
\begin{figure}[h]
	\begin{center}
	\includegraphics[width=0.9\linewidth]{ex3_force.eps} 
	\end{center}
	\caption{軸方向の外力を受ける棒部材.(a),(b)は集中荷重を,(c)は分布荷重を受ける
	部材を表す.(c)-iからiiiは,部材各点における分布荷重の大きさを表すグラフ.} 
	\label{fig:fig3}
\end{figure}
\subsection{境界値問題としての軸力問題}
応力$\sigma$とひずみ$\varepsilon$が, 比例係数$E$を用いて
\begin{equation}
	\sigma(x) =E(x) \varepsilon (x)
	\label{eqn:Hooke}
\end{equation}
と表されるような物体はフック固体と呼ばれ, 式(\ref{eqn:Hooke})の関係は
フック則を表す. 式(\ref{eqn:Hooke})を応力の釣り合い式(\ref{eqn:sigx_equib})に代入し, 
ひずみと変位の関係(\ref{eqn:e_dudx})を用いれば, 
変位に関する次の2階常微分方程式:
\begin{equation}
	\frac{d}{dx}\left( EA \frac{du}{dx} \right)+p=0
	\label{eqn:gvn_eq}
\end{equation}
が得られる. この微分方程式を適切な境界条件(支持条件)の元で解けば, 
変位分布が求められ, その結果を微分することでひずみ$\varepsilon$や
応力$\sigma$, 軸力$N$が得られる.

棒部材の支持条件(境界条件)には種々のものが考えられるが,以下のタイプが最も基本的である。
\begin{itemize}
\item
変位境界(固定端を含む): $u(b)=\bar{u}$\\
	($x=b$は境界位置の座標を, $\bar u$は与えられた変位量を表す。
	$\bar u=0$は固定端に相当する)
\item
荷重境界(自由端を含む):$N(b)=\bar{N}$\\
	 ($x=b$は境界位置の座標を, $\bar N$は与えられた外力を表す.
	$\bar{N}$は, 引張を正とする. $\bar N=0$は自由端に相当する)
\end{itemize}
断面剛性$EA$が場所によらず一定のとき,式(\ref{eqn:gvn_eq})は
\begin{equation}
	EA\frac{d^2u}{dx^2}+p=0
	\label{eqn:gvn_eq2}
\end{equation}
と,定数係数の2階線形常微分方程式となる.
\paragraph{問題}
図\ref{fig:fig4}に示す(a)から(c)のような棒部材について, 
変位と軸力分布を求め, その結果をグラフとして示せ.また, 各々の場合について, 
支点反力を求めよ.ただし, 図中に示した$p(x)$は分布力を意味し,断面剛性
$EA$は場所によらず一定とする.
\begin{figure}[h]
	\begin{center}
	\includegraphics[width=0.9\linewidth]{ex4_ODE.eps} 
	\end{center}
	\caption{軸方向の外力を受ける棒部材.$p=p(x)$は分布力を,$p_0$は与えられた定数を意味し, 
	断面剛性$EA$は場所によらず一定とする.} 
	\label{fig:fig4}
\end{figure}
%%%%%%%%%%%%%%%%%
%%%%%%%%%%%%%%%%%
\clearpage
\subsection{補足}
%\subsubsection{接頭辞}
\begin{table}[h]
\begin{center}
\caption{よく利用される数値に関する接頭辞(metric prefix)}
\label{tbl:prefix}
\begin{tabular}{c|c|c|c|c|c|c|c|c|c|c}
f & p & n & $\mu$ & m & & k & M & G & T & P \\
\hline
フェムト & ピコ& ナノ & マイクロ & ミリ & & キロ & メガ& ギガ& テラ & ペタ\\
\hline
$10^{-15}$ &
$10^{-12}$&
$10^{-9}$ &
$10^{-6}$ &
$10^{-3}$ &
$10^0$ &
$10^3$ &
$10^6$ &
$10^9$ &
$10^{12}$&
$10^{15}$ 
\end{tabular}
\end{center}
\end{table}
%\subsubsection{ギリシャ文字}
\begin{table}[h]
\begin{center}
\caption{ギリシャ文字とその読み方}
\begin{tabular}{c|c|c|c}
大文字 & 小文字 & 読み方(英語) &読み方(日本語)\\
\hline \hline
$A$ & $\alpha$ & alpha & アルファ \\
\hline
$B$ & $\beta$ & beta & ベータ \\
\hline
$\Gamma$ & $\gamma$ & gamma & ガンマ\\
\hline
$\Delta$ & $\delta$ & delta & デルタ\\
\hline
$E$ & $\epsilon, \varepsilon$ & epsilon & イプシロン \\
\hline
$Z$ & $\zeta$ & zeta & ゼータ \\
\hline
$H$ & $\eta$ & eta & イータ \\
\hline
$\Theta$ & $\theta$ & theta & シータ \\
\hline
$I$ & $\iota$ & iota & イオタ \\
\hline
$K$ & $\kappa$ & kappa & カッパ\\
\hline
$\Lambda$ & $\lambda$ & lambda & ラムダ \\
\hline
$M$ & $\mu$ & mu & ミュー \\
\hline
$N$ & $\nu$ & nu & ニュー\\
\hline
$\Xi$ & $\xi$ & xi & グザイ\\
\hline
$O$ & $o$ & omicron & オミクロン\\
\hline
$\Pi$ & $\pi,\varpi$ & pi & パイ \\
\hline
$P$ & $\rho$ & rho & ロー \\
\hline
$\Sigma$ & $\sigma$ & sigma & シグマ\\
\hline
$T$ & $\tau$ & tau &タウ \\
\hline
$\Upsilon$ & $\upsilon$ & upsilon & ウプシロン\\
\hline
$\Phi$ & $\phi, \varphi$ & phi &ファイ \\
\hline
$X$ & $\chi$ & chi & カイ \\
\hline
$\Psi$ & $\psi$ & psi & プサイ \\
\hline
$\Omega$ & $\omega$ & omega & オメガ\\
\end{tabular}
\end{center}
\end{table}
%%%%%%%%%%%%%%%%%%%%%%%%
\end{document}
%\begin{figure}[here]
\begin{figure}
	\vspace{-3mm}
	\begin{center}
	\includegraphics[width=0.45\linewidth]{fig1.eps} 
	\end{center}
	\vspace{-5mm}
	\caption{一端を固定壁に支持された棒部材とそのひずみ分布.} 
	\label{fig:fig1}
\end{figure}
%%%%%%%%%%%%%%%%%%%%%%%%%%%%%%%%%%%%%%%%%%%%

