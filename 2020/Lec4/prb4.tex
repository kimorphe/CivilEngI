\documentclass[10pt,a4j]{article}
%\usepackage{graphicx,wrapfig}
\usepackage{graphicx}
%\usepackage{showkeys}
\setlength{\topmargin}{-1.5cm}
%\setlength{\textwidth}{16.5cm}
\setlength{\textwidth}{14.5cm}
\setlength{\textheight}{25.2cm}
\newlength{\minitwocolumn}
\setlength{\minitwocolumn}{0.5\textwidth}
\addtolength{\minitwocolumn}{-\columnsep}
%\addtolength{\baselineskip}{-0.1\baselineskip}
%
\def\Mmaru#1{{\ooalign{\hfil#1\/\hfil\crcr
\raise.167ex\hbox{\mathhexbox 20D}}}}
%
\begin{document}
\newcommand{\fat}[1]{\mbox{\boldmath $#1$}}
\newcommand{\D}{\partial}
\newcommand{\w}{\omega}
\newcommand{\ga}{\alpha}
\newcommand{\gb}{\beta}
\newcommand{\gx}{\xi}
\newcommand{\gz}{\zeta}
\newcommand{\vhat}[1]{\hat{\fat{#1}}}
\newcommand{\spc}{\vspace{0.7\baselineskip}}
\newcommand{\halfspc}{\vspace{0.3\baselineskip}}
\bibliographystyle{unsrt}
%\pagestyle{empty}
\newcommand{\twofig}[2]
 {
   \begin{figure}
     \begin{minipage}[t]{\minitwocolumn}
         \begin{center}   #1
         \end{center}
     \end{minipage}
         \hspace{\columnsep}
     \begin{minipage}[t]{\minitwocolumn}
         \begin{center} #2
         \end{center}
     \end{minipage}
   \end{figure}
 }
%%%%%%%%%%%%%%%%%%%%%%%%%%%%%%%%%
%\vspace*{\baselineskip}
\begin{flushright}
	{Civil Engineering I \\ 
	Due: 23:59, May 14,2020
	}
\end{flushright}
%%%%%%%%%%%%%%%%%%%%%%%%%%%%%%%%%%%%%%%%%%%%%%%%%%%%%%%%%%%%%%%%
\hspace{10mm}
\section*{Assignment 4}
Consider a simply supported truss shown in Fig.\ref{fig:prb} consisting 
of circular members of equal and uniform cross-sectional diameter $d$. 
For this truss structure, determine the minimum value for $d$ following 
the same procedure given in lecture note 4. Compare the truss designed 
in the lecture note with that of Fig.\ref{fig:prb}, and discuss which 
design is better in what respect(s) using numerical values given 
in the lecture note. The assumptions in the lecture note are all 
adopted here as well. 
    \begin{figure}[h]
        \begin{center}
        \includegraphics[width=1.0\linewidth]{prb.eps} 
        \end{center}
        \caption{
	A simply supported truss to cross the pipes over a trench of width $3l$.
	  } 
        \label{fig:prb}
    \end{figure}
\end{document}
