\documentclass[10pt,a4j]{article}
%\usepackage{graphicx,wrapfig}
\usepackage{graphicx}
\usepackage{showkeys}
\setlength{\topmargin}{-1.5cm}
%\setlength{\leftmargin}{1.5cm}
%\setlength{\textwidth}{15.5cm}
\setlength{\textheight}{25.2cm}
\newlength{\minitwocolumn}
\setlength{\minitwocolumn}{0.5\textwidth}
\addtolength{\minitwocolumn}{-\columnsep}
%\addtolength{\baselineskip}{-0.1\baselineskip}
%
\def\Mmaru#1{{\ooalign{\hfil#1\/\hfil\crcr
\raise.167ex\hbox{\mathhexbox 20D}}}}
%
\begin{document}
\newcommand{\fat}[1]{\mbox{\boldmath $#1$}}
\newcommand{\D}{\partial}
\newcommand{\w}{\omega}
\newcommand{\ga}{\alpha}
\newcommand{\gb}{\beta}
\newcommand{\gx}{\xi}
\newcommand{\gz}{\zeta}
\newcommand{\vhat}[1]{\hat{\fat{#1}}}
\newcommand{\spc}{\vspace{0.7\baselineskip}}
\newcommand{\halfspc}{\vspace{0.3\baselineskip}}
\bibliographystyle{unsrt}
%\pagestyle{empty}
\newcommand{\twofig}[2]
 {
   \begin{figure}
     \begin{minipage}[t]{\minitwocolumn}
         \begin{center}   #1
         \end{center}
     \end{minipage}
         \hspace{\columnsep}
     \begin{minipage}[t]{\minitwocolumn}
         \begin{center} #2
         \end{center}
     \end{minipage}
   \end{figure}
 }
%%%%%%%%%%%%%%%%%%%%%%%%%%%%%%%%%
%\vspace*{\baselineskip}
\begin{flushright}
	Civil Engineering I \\
	May 7, 2020
\end{flushright}
\begin{center}
	{\Large \bf Lecture Note 3 }
\end{center}
\setcounter{section}{2}
\section{Basics of Truss Structures}
\subsection{Introduction}
A bridge structure is an assembleage of plates, bars, and cables.
The parts making up the bridge structure are called structural members.
In this class, we're going to focus on slender bar-like structural members. 
By "bar", we mean a slender structural member for which an evident longitudinal 
axis exists. The cross sectional area of the bar can either be uniform or 
nonuniform in the longitudinal axis. By joining the bar-like members, we can 
build a structure called truss. Therefore, we will call the bar-like structural 
member as a truss member. Fig.\ref{fig:fig1} shows examples of truss stuctures. 
The truss structures shown in this figures bear the name of their inventors. 
Truss structures are characterized by a triangular assemblage of the members. 
The major advantage of the truss structure is the weight. 
Truss structures are often lighter in weight than girder bridges, which 
is the most prevalent bridge structure best suited to short-spanned bridges.
In designing truss structures, the first questions to be answered are
\begin{itemize}
\item
	How is the whole strctures supported ?
\item
	How to connect two or more structural members at the joints ?
\item
	How much force each truss member has to bear ?
\end{itemize}
\begin{figure}[h]
	\begin{center}
	\includegraphics[width=1.0\linewidth]{fig1.eps} 
	\end{center}
	\caption{Examples of truss structures. (a)Warren truss, (b) Howe truss, and (c) Pratt truss.} 
	\label{fig:fig1}
\end{figure}
Those are the issues discussed in this class. 
\subsection{Support Condition and Reaction Forces}
There are three support condtions that are basic in structural mechanics. 
Those are 
\begin{itemize}
\item
	fixed(or clamped) support,
\item
	pin( or hinge) support,and 
\item
	roller support.
\end{itemize}
In the followings, the three suppport conditions are explained
looking at the constraints that each support imposes to the 
body being supported. To this end, we do not have to be specific about the 
geometry nor the size of the supported objects. Therefore, we're going to 
study the support conditions for a single bar without loosing generaity.
\subsubsection{Fixed support}
When a bar is fixed perfectly to a rigid and immobile wall as shown in Fig.\ref{fig:fig3}-(a), 
we say the bar is connected to a "fixed" or "clamped" support. 
At the fixed support, the translational and rotational motion are completely restraied. 
In the present example, the horizontal, vertiacl, and the rotational motion at the left 
end of the bar is completely fixed.
To see the action of the fixed support on the bar, suppose that the bar is loaded with
a vertical force as illustrated in Fig.\ref{fig:fig3}-(a).
The support (wall) should apply the forces and/or moment to the bar to restrain the 
dispalacement at $A$ and to hold the bar as in Fig.\ref{fig:fig3}-(a). 
Such forces supplied by supports in response to the loading is called reaction forces. 
To see this more closely, assume that the bar is subjected to the reaction forces $V,H$ 
and moment $M$ at A (see. Fig.\ref{fig:fig3}-(b)).
Then the equilibrium condtions for horizontal and vertical forces conclude the following. 
\begin{equation}
	H=0, \ \ F-V=0 \ \ \Rightarrow V=F
	\label{eqn:}
\end{equation}
If we evaluate the total moment about $A$, the equilibrium condition for the moment  
 is written as 
\begin{equation}
	M+F\times l =0 \ \ \Rightarrow M=-Fl.
	\label{eqn:}
\end{equation}
We have not studied the way to impose moment without applying forces.
However, a pure moment of arbitrary magnitude can be generated by 
a couple forces shown in Fig.\ref{fig:fig6}. Note that the moment at the 
center of the couple force is evaluated to be 
\begin{equation}
	M=K\varepsilon, 
	\label{eqn:}
\end{equation}
while the sum of the coupled forces is zero. 
Thus we may generate the moment of arbitray magnitude by selecting 
 either $K$ or $\varepsilon$ withouth violating the equilbrium of forces. 
 Note also that we cannot determine both $F$ and $\varepsilon$ simultaneouslu, thus 
we should content with having only the moment. 
This is why we have drawn the reaction moment in Fig.\ref{fig:fig3} but not the couple force.
\begin{figure}[h]
	\begin{center}
	\includegraphics[width=0.8\linewidth]{fig3.eps} 
	\end{center}
	\caption{(a) A bar fixed horizontally to a rigid and immobile wall. 
	The reaction forces from the wall.} 
	\label{fig:fig3}
\end{figure}
\begin{figure}[h]
	\begin{center}
	\includegraphics[width=0.4\linewidth]{fig6.eps} 
	\end{center}
	\caption{Generation of a pure moment by a couple force.} 
	\label{fig:fig6}
\end{figure}
\subsubsection{Pin support}
The pin support fixes a structural member with a hinge around which the body may 
rotate frictionlessly. Since the support itself is fixed to an immobile floor 
or a wall, the displacement is not allowed in any direction at the support. 
This implies that the pin support applies the horizontal and vertical reaction 
forces in response to the external loading to the supported object. 
On the other hand, the reaction moment does not arise because the support would 
not prevent the object from rotatating at the hinge. 
\begin{figure}[h]
	\begin{center}
	\includegraphics[width=0.6\linewidth]{fig7.eps} 
	\end{center}
	\caption{Pin supports installed on immobile floor and wall.} 
	\label{fig:fig7}
\end{figure}
\subsubsection{Roller support}
The roller support is considered as a hinge support mounted on a roller 
that can slide freely on a floor or a wall. 
It works in the same as the hinge support as far as the rotational motion 
 and the reaction moment are concerened. The roller support allows the supported object 
 to rotate without generateing a reaction moment. Moreover, it allows the displacement 
 in the direction that the roller may slide freely. If roller support is installed to a wall, 
 the vertical dipacelement is allowed while the horizontal is not. 
 In this case, therefore, the support can be a source of only the horizontal reaction force.
When the roller support is installed on the horizontal floor, the support 
 constrains the vertical motion while letting the horizontal displacement take place freely. 
This means the reaction force directs upward or downward and the horizontal 
reaction force is nonexistent.
\begin{figure}[h]
	\begin{center}
	\includegraphics[width=0.6\linewidth]{fig8.eps} 
	\end{center}
	\caption{Roller supports installed on immobile floor and wall.} 
	\label{fig:fig8}
\end{figure}
\subsubsection{Simple support}
When a bar is supported by a pin support at one end and by a roller 
support by the other end, we say the bar is simply supported. 
The combination of the roller and pin support is called simple support. 
When a bar is simply supported, we can determine the reaction forces 
from the equilibirium conditions for the force and moment. 
To see this, consider a simply supported bar shown in Fig.\ref{fig:fig3_2}
subjected to an angled load.
The pin support at the left end applies both the vertical and horizontal reaction forces 
since the motion except rotation is constrained at the left end.
At the right end of the bar, the rotation and lateral motion are allowed.
Only a vertical reaction force arises at the right end. 
Fig.\ref{fig:fig3_1} shows the non-vanishing reaction forces with the external 
force at the center of the bar. The diagram showing the forces acting to a body or 
a portion of the body is called free body diagram. Fig.\ref{fig:fig3_1} is an example 
of the free body diagram for a simply supported bar.
\begin{figure}[h]
	\begin{center}
	\includegraphics[width=0.6\linewidth]{fig3_2.eps} 
	\end{center}
	\caption{A simply supported bar AC subjected to an oblique force of magnitude $F$.}
	\label{fig:fig3_2}
\end{figure}
\begin{figure}[h]
	\begin{center}
	\includegraphics[width=0.6\linewidth]{fig3_1.eps} 
	\end{center}
	\caption{
		A free body diagram for a simply supported bar.
		$F, H_A, V_A, V_C$ and $|\fat{f}|=F$ denote the magnitude of forces.
	}
	\label{fig:fig3_1}
\end{figure}
The equilbirium conditon for the moment about $A$ is 
\[
	-F\sin\frac{\pi}{4}\times \frac{l}{2} + V_C\times l =0,
\]
Thus we have $V_C=\frac{F}{2\sqrt{2}}$. The equilibrium of the horizontal force 
gives us 
\[
	H=F\cos\frac{\pi}{4}=\frac{F}{\sqrt{2}}.
\]
Finally, the equilbirium of vertical forces requires 
\[
	F\sin\frac{\pi}{4} -(V_A+V_C)=0
\]
determines $V_A$ as 
\[
	V_A=\frac{F}{2\sqrt{2}}.
\]
\subsection{Pin Connection}
Two or more truss members are usually connected either by a gasset plate or a hinge. 
The former type of connection is called a rigid connection where as the latter 
is called a pin connection.
At a rigid connection, the truss members are joined rigidly by a gasset plate.
This type of connection is considered rigid because the truss members are joined 
titely by the gasset plate that the angles between the truss members do not 
change upon loading. 
This means that the joint can transmitt a moment from one member to the other. 
At pin connections, on the other hand, the truss members joined by the 
hinges may rotate freely abount the joint. 
Thus the pin connected joint does not support a moment, and it does not 
transmitt the moment from one member to the other. 
The consequence is important in the analyses of truss structure. 

If we look at a truss member with rigidly connected ends, 
the force and the moments work on both ends. This is the case 
even thought the trusss member is not directly loaded by external forces. 
When the structure and the structural members are in static equilibrium, the 
equilibrium condtions must be met, which requires
\begin{equation}
	N_A-N_B=0, \ Q_A-Q_B=0, \  M_A-M_B+Q_Bl=0 
	\label{eqn:}
\end{equation}
\begin{equation}
\end{equation}
and thus we have
\begin{equation}
	N_A=N_B=N,\  Q_A=Q_B=Q, \  M_A-M_B=Ql.
	\label{eqn:sec_forces}
\end{equation}
where $N$ and $Q$ are called axial and shear force, respectively.
When the truss member is connected to pins at both ends, then 
$M_A=M_B=0$ by assumption. This reduces to eq.(\ref{eqn:sec_forces}) 
to 
\begin{equation}
	N_A=N_B=N,\  Q_A=Q_B=0, \  M_A=M_B=0.
	\label{eqn:sec_forces}
\end{equation}
This means that the truss members transmit only the axial forces 
 if they are connected at both ends by hinges. As we will see, this 
 greatly simplifies the analyses of truss structures. 
\begin{figure}[h]
	\begin{center}
	\includegraphics[width=0.8\linewidth]{fig4.eps} 
	\end{center}
	\caption{Rigid and pin connections joining two truss members.} 
	\label{fig:fig4}
\end{figure}
\begin{figure}[h]
	\begin{center}
	\includegraphics[width=0.8\linewidth]{fig5.eps} 
	\end{center}
	\caption{The axial force transmitted to a truss member.}
	\label{fig:fig5}
\end{figure}
%
%
\subsection{Analysis of Truss Structure}
One of the primary objectives of the analysis of truss structure is to determine 
the axial forces in truss members generated due to  the extrnal load. 
Evaluation of the axial forces is important because the excessive axial force leads to the 
failure of the member and/or the structure.
When a pin-connected truss is simply supported, the reaction forces from the supports and the 
axial forces in the truss members can be determined from the equilbrium conditions of forces. 
To obtain the axial forces, we're going to impose equilbrium conditions node by node.   
There are two other approaches to this problem. However the node-by-node method is the simplest. 
To study this approach, consider determining axial forces in the truss shown in 
Fig.\ref{fig:fig9}. This is one of the simplest truss structures of all, but it can serve as a 
canonical problem in learning the node-by-node method.\\ 


\begin{figure}[h]
	\begin{center}
	\includegraphics[width=0.4\linewidth]{fig9.eps} 
	\end{center}
	\caption{A simply supported truss.} 
	\label{fig:fig9}
\end{figure}
The analysis of truss structure usually start with the identification of 
external forces that act to the structure. 
The external loads include reaction forces.  
In the present problem, the horizontal force at node $C$ of magnitude $F$ 
is the given external load.
The reaction forces are not known, but we can determine them using 
equilbrium conditions for the whole truss structure. To write the 
equilbrium equations, we draw a free body diagram as shown in Fig.\ref{fig:fig9}-(a)
where $H_A, V_A$ and $V_B$ are the horizontal or vertical component 
of the reaction forces. The arrows in this figure indicate the direction of 
forces when the vector componets take positive values. 
With the foregoing notation, the equilbrium equations of the forces 
and moment are written as
\begin{equation}
	F+H_A=0, \ \ V_A+V_B=0, \ \ -F\times l\sin\frac{\pi}{3}+V_B\times l=0.
	\label{eqn:}
\end{equation}
Note that the total moment about $A$ has been considered in the third equation.
Thus, we have 
\begin{equation}
	H_A=-F, \ \ V_A=-V_B=\frac{\sqrt{3}}{2}F
	\label{eqn:}
\end{equation}
Once we have obtained every reaction force, we're ready to determine the axial forces. 
For the analysis of the axial forces, it is convenient to enumerate the truss members as 
shown in Fig.\ref{fig:fig8}-(a), and denote the axial forces of the $i$th truss member 
as $N_i$ for $i=1,2$ and 3. Note that the tensile axial force is taken to be positive. 
If a tensile axial force is acting to a truss member, then the truss member is 
being pulled by the joints it is connected to. 
In other words, the truss member under a tensile axial force is pulling 
the joints at its end. This is a consequence of the law of action and reaction.
In the present situation, body A is the truss member and body B is the joint connected to.
To make things clearer, let's look at the truss member 1. The truss member 1 is connected 
to node A and B. Therefore, it is pulling the nodes A and B together if $N_1>0$. 
The same observation applies to the truss members 2 and 3. 
Figure.\ref{fig:fig9}-(b) shows the forces that act to each node (joints) of the truss. 
It should be noted that this figure depicts the forces acting to the node not the truss members.  
Since the truss structure including its joints is at rest, the set of forces 
applied to each joint must satisfy the equilbrium condition.
By imposing the equilbrium conditions for horizontal and vertical forces node by node, 
we can determine the axial forces. This procedure goes as follows.
\begin{enumerate}
\item
	Node A: \\
	The equilbrium equations of the horizontal and vertical forces at node A 
	are written respectively as 
	\begin{equation}
		N_1+H_A+N_2\cos\alpha=0, \ \ N_2\sin\alpha+V_A=0
		\label{eqn:nodeA}
	\end{equation}
	where $\alpha=\frac{\pi}{3}$. Solving eqs.(\ref{eqn:nodeA}), 
	we have $N_1=\frac{F}{2}$ and $N_2=F$. 
\item
	Node B: \\
	It follows from the equiribrium condtions for node B written as 
	\begin{equation}
		N_1+N_3\cos\alpha=0, \ \ N_3\sin\alpha+V_B=0
		\label{eqn:nodeB}
	\end{equation}
	that we have $N_1=\frac{F}{2}$ and $N_3=-F$.
	We can verify that the axial force for member 1 from eqs.(\ref{eqn:nodeA}) 
	agrees with that from eq.(\ref{eqn:nodeB}).
\item
	Node C: \\
	The equilbrium conditions for node C are written as 
	\begin{equation}
		F-N_1\cos\alpha+N_3\cos\alpha=0, \ \ N_2\sin\alpha+N_3\sin\alpha=0.
		\label{eqn:nodeC}
	\end{equation}
	Since we have already determined all the axial forces, 
	eqs.(\ref{eqn:nodeC}) are redundant.  
	However, it is a good practice to verify the solutions using a redundant 
	but independently derived relations that should be satisfied as well.  
	In the present problem, we can verify that our solutions $N_1,N_2$ and $N_3$ do 
	satisfy eq.(\ref{eqn:nodeC}).
\end{enumerate}
As it has been shown, two equations are availablle for every nodal point.
This means we can determine up to two axial forces per node 
based on the  equilbrium conditions. 
If more than two members are connected to a node, it is not possible to 
determine the axial forces of those members all at once as long as 
the nodal equilrium conditions are used exlusively.
\begin{figure}[h]
	\begin{center}
	\includegraphics[width=0.8\linewidth]{fig10.eps} 
	\end{center}
	\caption{(a) A free body diagram for a truss of Fig.\ref{fig:fig9}. 
	(b) The forces exerted on the nodal points by the truss members.} 
	\label{fig:fig10}
\end{figure}
\subsection{Solved Problems}
\begin{enumerate}
\item
Consider a straight bar AB held by a fixed support as shown in Fig.\ref{fig:fig3_3}.
Determine the reaction forces and reaction moment at the fixed end $A$ when 
a point force of magnitude $F$ is applied obliquely to the right end $B$ 
of the bar.  
\begin{figure}[h]
	\begin{center}
	\includegraphics[width=0.6\linewidth]{fig3_3.eps} 
	\end{center}
	\caption{A bar connected to a fixed support. 
	An oblique load of magnitude $F$ is applied to the right end of the bar.} 
	\label{fig:fig3_3}
\end{figure}
\item
Examine the support conditions shown in Fig.\ref{fig:fig3_4}, and find whether 
all the reaction forces and moments can be determined from the equilibrium equations. 
\begin{figure}[h]
	\begin{center}
	\includegraphics[width=0.8\linewidth]{fig3_4.eps} 
	\end{center}
	\caption{A straight bar connected to (a)fixed supports, and (b) pinned supports.} 
	\label{fig:fig3_4}
\end{figure}
\item
	Determine the reaction and axial forces acting to the simply supported 
	truss shown in Fig.\ref{fig:13}.
\begin{figure}[h]
	\begin{center}
	\includegraphics[width=0.5\linewidth]{fig13.eps} 
	\end{center}
	\caption{A simply supported truss structure.} 
	\label{fig:fig13}
\end{figure}
\begin{figure}[h]
	\begin{center}
	\includegraphics[width=1.0\linewidth]{fig14.eps} 
	\end{center}
	\caption{(a)A free body diagram of the whole truss structure.
	(b) Axial forces acting to the joints connected by hinges.} 
	\label{fig:fig14}
\end{figure}
\end{enumerate}
\subsection{Supplementary Problems}
\begin{enumerate}
\item
Consider a bar subjected to two angled forces as shown in Fig.\ref{fig:fig3_5}, and 
determine the reaction forces and moment at the fixed support.
\begin{figure}[h]
	\begin{center}
	\includegraphics[width=0.5\linewidth]{fig11.eps} 
	\end{center}
	\caption{A bar subjected to two angled forces of magnitude $F$.}
	\label{fig:fig11}
\end{figure}
\item
	Examine the bars shown in Fig.\ref{fig:fig12} under a vertical load 
	of magnitude $F$, and answer whether the reaction forces and 
	moment can be determined from the equilbrium conditions.
	If the answer is positive, obtain the reaction forces and moment. 
\begin{figure}[h]
	\begin{center}
	\includegraphics[width=1.0\linewidth]{fig12.eps} 
	\end{center}
	\caption{Bars subjected to a vertical force of magnitude $F$.} 
	\label{fig:fig12}
\end{figure}
\item
Determine the reaction and axial forces acting to the truss structures shown in Fig.\ref{fig:fig3_5}.
\begin{figure}[h]
	\begin{center}
	\includegraphics[width=1.0\linewidth]{fig3_5.eps} 
	\end{center}
	\caption{Simply supported truss structures.} 
	\label{fig:fig3_5}
\end{figure}
\end{enumerate}
\end{document}
%%%%%%%%%%%%%%%%%%%%%%%%%%%%%%%%%%%%%%%%%%%%%%%%%%%%%%%%%%%%%%%%
