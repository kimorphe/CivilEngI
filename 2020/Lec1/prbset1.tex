Exercise 2-1
Obtain the moments due to the downward vertical force $\fat{f}$ shown in fig.\ref{fig:fig1} 
with respect to the points $P_1,P_2,\dots P_4$, respectively.

Exercise 2-2
Obtain the moments due to the horizontal force $\fat{f}$ shown in fig.\ref{fig:fig2} 
with respect to the points $P_1,P_2,\dots P_4$, respectively.

Exercise 2-3
Obtain the moments due to the force $\fat{f}$ shown in fig.\ref{fig:fig_3} 
with respect to the points $P_1,P_2,\dots P_5$, respectively.

Exercise 2-4
Obtain the moments due to the force $\fat{f}$ shown in fig.\ref{fig:fig_4} 
with respect to the points $P_1,P_2,\dots P_5$, respectively.

Exercise 2-5
Determine the magnitude of the force $\fat{f}$ shown in fig.\ref{fig:fig_5} 
that is necessary to equilibrate the bar. 

Exercise 2-5
Obtain the total force and the total moments about $P_1, P_2,\dots P_5$, 
and show that the bar shown in fig.\ref{fig:fig_5} is in equilibrium 
in terms of both force and moments. 
Note that $F$ and $F/2$ mean the magnitude of forces applied vertically to the bar.




   
