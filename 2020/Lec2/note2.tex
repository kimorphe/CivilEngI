\documentclass[10pt,a4j]{article}
%\usepackage{graphicx,wrapfig}
\usepackage{graphicx}
\setlength{\topmargin}{-1.5cm}
%\setlength{\leftmargin}{1.5cm}
%\setlength{\textwidth}{15.5cm}
\setlength{\textheight}{25.2cm}
\newlength{\minitwocolumn}
\setlength{\minitwocolumn}{0.5\textwidth}
\addtolength{\minitwocolumn}{-\columnsep}
%\addtolength{\baselineskip}{-0.1\baselineskip}
%
\def\Mmaru#1{{\ooalign{\hfil#1\/\hfil\crcr
\raise.167ex\hbox{\mathhexbox 20D}}}}
%
\begin{document}
\newcommand{\fat}[1]{\mbox{\boldmath $#1$}}
\newcommand{\D}{\partial}
\newcommand{\w}{\omega}
\newcommand{\ga}{\alpha}
\newcommand{\gb}{\beta}
\newcommand{\gx}{\xi}
\newcommand{\gz}{\zeta}
\newcommand{\vhat}[1]{\hat{\fat{#1}}}
\newcommand{\spc}{\vspace{0.7\baselineskip}}
\newcommand{\halfspc}{\vspace{0.3\baselineskip}}
\bibliographystyle{unsrt}
%\pagestyle{empty}
\newcommand{\twofig}[2]
 {
   \begin{figure}
     \begin{minipage}[t]{\minitwocolumn}
         \begin{center}   #1
         \end{center}
     \end{minipage}
         \hspace{\columnsep}
     \begin{minipage}[t]{\minitwocolumn}
         \begin{center} #2
         \end{center}
     \end{minipage}
   \end{figure}
 }
%%%%%%%%%%%%%%%%%%%%%%%%%%%%%%%%%
%\vspace*{\baselineskip}
\begin{center}
	{\Large \bf Lecture Note 2, Civil Engineering I \\
	- introduction to structural mechanics-\\
	Equilibrium of Force and Moment
	}  \\
\end{center}
\vspace{1.5cm}
\setcounter{section}{1}
\section{Equilbrium}
\begin{figure}[h]
	\begin{center}
	\includegraphics[width=1.0\linewidth]{fig1.eps} 
	\end{center}
	\caption{Three equivalent forces in terms of the moment about pivot P.} 
	\label{fig:fig1}
\end{figure}
\section*{Excersises 2}
\begin{itemize}
\item
Exercise 2-1:\\
Obtain moments with respect to points $P_1,P_2,\dots P_4$ due to the downward 
vertical force $\fat{f}$ shown in Fig.\ref{fig:fig2_1} -(a).
\item
Exercise 2-2:\\
Obtain moments with respect to points $P_1,P_2,\dots P_4$ due to 
the horizontal force $\fat{f}$ shown in Fig.\ref{fig:fig2_1}-(b). 
\item
Exercise 2-3:\\
Obtain moments with respect to points $P_2,P_3,\dots P_5$
due to the load of magnitude $F$ acting obliquely to point $P_1$ 
as shown in Fig.\ref{fig:fig2_2}-(a).
\item
Exercise 2-4:\\
Obtain moments with respect to points $P_1,P_2,\dots P_4$
 due to the force $\fat{f}$ acting obliquely to point $P_5$ as 
		shown in Fig.\ref{fig:fig2_2}-(b).
\item
Exercise 2-5:\\
	Determine the magnitude of the force $\fat{f}$ shown in 
	Fig.\ref{fig:fig2_3}-(a) that is necessary to equilibrate 
	the bar both in terms of the force and moment. 
\item
Exercise 2-6:\\
	Calculate the total force and the total moments about $P_1, P_2,\dots P_5$, 
	to show that the bar shown in Fig.\ref{fig:fig2_3}-(b) is in an equilibrium state 
	in terms of both the force and moment. 
	Note that $F$ and $F/2$ are the magnitude of the downward and upward forces 
	applied vertically to the bar, respectively.
\end{itemize}
\newpage
\begin{figure}[h]
	\begin{center}
	\includegraphics[width=0.7\linewidth]{fig2_1.eps} 
	\end{center}
	\caption{Vertical and horizontal forces acting on a body.} 
	\label{fig:fig2_1}
\end{figure}
\begin{figure}[h]
	\begin{center}
	\includegraphics[width=0.7\linewidth]{fig2_2.eps} 
	\end{center}
	\caption{Forces acting obliquely to a body.} 
	\label{fig:fig2_2}
\end{figure}
\begin{figure}[h]
	\begin{center}
	\includegraphics[width=0.8\linewidth]{fig2_3.eps} 
	\end{center}
	\caption{Straight bars subjected to a set of forces.} 
	\label{fig:fig2_3}
\end{figure}
\end{document}
%%%%%%%%%%%%%%%%%%%%%%%%%%%%%%%%%%%%%%%%%%%%%%%%%%%%%%%%%%%%%%%%
